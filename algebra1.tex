\documentclass[10pt,a4paper]{article}
\usepackage[utf8]{inputenc}
\usepackage[german]{babel}
\usepackage{mathtools}
\usepackage{mathtools}
\usepackage{amsmath}
\usepackage{amsfonts}
\usepackage{amssymb}
\usepackage{graphicx}
\usepackage[left=2cm,right=2cm,top=2cm,bottom=2cm]{geometry}

\usepackage{multicol} % Multiple columns

% Setting information for document
\author{Jonas Betzendahl}
\date{\today}

% For definitions
\newtheorem{defi}{Definition}
\newtheorem{satz}{Satz}
\newtheorem{korr}{Korrolar}

\begin{document}
\newcommand{\menge}[2]{$\{\,{#1}\,\vert\,{#2}\,\}$}
\parindent0pt

\begin{center}

\Huge \textbf{\underline{Algebra I}} \\\bigskip
\normalsize\normalsize Stand vom \today\\

\begin{multicols}{2}
Dozent: Prof. Dr. Christopher Voll\\
\texttt{voll@math.uni-bielefeld.de   }\bigskip

Autor: Jonas Betzendahl\\
\texttt{jbetzend@techfak.uni-bielefeld.de}\bigskip
\end{multicols}
\end{center}

\tableofcontents
\newpage

\setcounter{section}{-2} % Kompensieren für Organisatorisches und Einführung

\section{Organisatorisches etc.}

Dozent ist Prof. Dr. Christopher Voll\\
\texttt{voll@math.uni-bielefeld.de}\\
Büro: UHG V5-238, Sprechstunde noch im Flux\bigskip

Vorlesungen finden an Montagen von 08.30 Uhr bis 10.00 Uhr und Mittwochs von 14.15 Uhr bis 15.45 Uhr statt.\bigskip

Es wird darauf hingewiesen, dass die Übungen bei Dr. Doang in Englisch abgehalten werden.\bigskip

Voraussetzung für die Zulassung zur Prüfung sind das Erreichen von mindestens 50 \% der Punkte und mindestens zwei Mal eine aktive Teilnahme an den Übungen (Vorrechnen) abgeleistet zu haben.\bigskip

\begin{tabular}{rlc}
\textbf{Bücher:} & Einführung in die Algebra (F. Lorenz, Spektrum)&\\
        & Algebra 1 (S. Bosch, Springer)&\\
        & Algebra (S. Lang, Springer)&\\
        & Algebra (Hungerford)&\\
        & Algebra (v.d. Waerden) & (Ein Klassiker)\\
        & Algebra (E. Artin) &
\end{tabular}\bigskip

Übungszettel gibt es immer am Mittwoch der Woche n, bearbeiten werden müssen diese bis Mittwoch der Woche n+1 (Abgabe vor der Vorlesung im Postfach des Tutors), besprochen werden sie in der Woche n+2 in den Tutorien.
         
\section{Algebra - die Kunst, Gleichungen zu lösen (Einführung)}

\underline{Lineare} Algebra:

$a_{11} x_1 + a_{12} x_2 + \dots + a_{nm} x_m = b_1$\\
$ \dots $\\
$a_{n1} x_1 + a_{n2} x_2 + \dots + a_{nm} x_m = b_n$\\

Alle Fragen können in Form $a_{ij}, b_i \in \mathcal{K}$ beantwortet werden.

Verallgemeinerung: $a_0 + a_1 x + a_2 x^2 + \dots + a_n x^n = 0$ (polynom. Gleichung von Grad $n (a_n \neq 0)$.

\glqq Struktur\grqq\ der Lösungen solcher Gleichungen treibt Menschen seit Jahrtausenden um.

Spezialfall: Quadratische Gleichungen

$$ x^2 + bx + c = 0 \Leftrightarrow \frac{-b \pm \sqrt{b^2 - 4c}}{21}$$

durch Wurzeln lösbar. Kubische Gleichungen:

$$ x^3 + b x^2 + cx + d = 0 \Leftrightarrow \dots $$

ebenfalls durch Wurzeln lösbar. Im 16. Jahrhundert wurde bekannt, dass auch quartische Gleichungen $(n = 4)$ durch Wurzelausdrücke lösbar sind.

Was nicht ins Weltbild des 16. Jahrhundert passte: Im 19. Jahrhundert zeigte Abel: Nicht jede quintische Gleichung $(n = 5)$ kann durch Wurzelausdrücke gelöst werden.

Galois: Lösungen von Polynomen sind nicht einfach Mengen ohne Struktur sondern Mengen \emph{mit} Struktur ($\rightarrow$ Gruppentheorie). Auflösbarkeit von $f = 0$ durch Wurzel $\Rightarrow$ Galois-Gruppe($f$) auflösbar.

Ziel der Vorlesung: Einführung in die Sprache der modernen Algebra, sowohl durch Anerkennen der Theorie als auch durch das Praktizieren.

\newpage
\section{Fundamente der Gruppentheorie}

\subsection{Monoide \& Gruppen}

\begin{defi}
	Ein \emph{Monoid} ist eine Menge $M$ zusammen mit einer Verknüpfung $\cdot : M \times M \to M$,
	die die Eigenschaften erfüllt:
	\begin{itemize}
		\item \emph{\texttt{(ASS)}} $\forall a,b,c \in M: (a \cdot b) \cdot c = a \cdot (b \cdot c)$ (Assoziativität)
		\item \emph{\texttt{(NEU)}} $\exists\, e = e_M \in M \; \forall a \in M,\, e \cdot a = a = a \cdot e$ (Existenz
	eines neutralen Elementes)
	\end{itemize} 
\end{defi}

\textbf{Bemerkung:} Die Notation ist oft einfach nur \glqq $ab$\grqq\ statt \glqq $a \cdot b$\grqq , oft auch bei mehreren Monoiden gleichzeitig. Ausgelassen wird immer die passende Verknüpfung. Es wird auch die Schreibweise $\prod_{i=1}^{n} a_i$ für den Ausdruck $a_1 \cdot a_2 \cdot \dots a_n,\; a_i \in M$ verwendet. Weiterhin gelten per Konvention: $\prod_{i=1}^{n} a_i = e$ für $n \leq 0$ und $a^m = \underbrace{a \cdot a \cdot \dots a}_{\text{m-mal, } m \in \mathbb{N}}$.

\textbf{Behauptung:} $e \in M$ ist eindeutig (siehe unten).\bigskip

Sei $a \in M$. Wir nennen $b \in M$ \emph{invers zu} $a$ falls $a \cdot b = b \cdot a = e$ gilt. Falls (!) solch ein $b$ existiert, ist es eindeutig (siehe unten). In diesem Fall ist die Notation oft $a^{-1}$ für $b$: $a \cdot a^{-1} = a^{-1} \cdot a = e$.\bigskip

\begin{defi} Eine \emph{Gruppe} ist ein Monoid $(G, \cdot)$ mit der folgenden Eigenschaft: 

\begin{center}
\emph{\texttt{(INV)}} $\forall a \in G\; \exists\, a^{-1} \in G$ sodass $a \cdot a^{-1} = a^{-1} \cdot a = e$ (Existenz eines inversen Elements)
\end{center}
$G$ heißt \emph{kommutativ} oder synonym dazu \emph{abelsch} falls folgende Eigenschaft gilt:
\begin{center}
\emph{\texttt{(KOM)}} $\forall a, b \in G: a \cdot b = b \cdot a$ (Kommutativität)
\end{center}
\end{defi}

Vektorräume zum Beispiel sind abelsche Gruppen, die interessante Struktur ist hier aber nicht die abelsche Eigenschaft sondern die Multiplikation mit Skalaren, die sich gut mit der Gruppenstruktur verträgt.\bigskip

% Porträt: N.H. Abel 1902-1829 (Norweger).

Die \emph{Ordnung} einer Gruppe ist $(G, \cdot)$ ist die Kardinalität $\vert G \vert$ von $G$.\bigskip

Konvention: \glqq Gruppe $G$\grqq , falls $\cdot$ klar ist. Ist $G$ abelsch, so schreibt man oft $+$ für $\cdot$ (die Monoidverknüpfung) und man redet von \glqq additiver Schreibweise\grqq\ im Gegensatz zu \glqq multiplikativer Schreibweise\grqq . Bei additiver Schreibweise schreibt man oft $0$ für $e$, bzw. bei multiplikativer Schreibweise $1$.\bigskip

\textbf{Beispiele 1.3:}
\begin{enumerate}
	\item $(\mathbb{Z}, +), (\mathbb{Q}, +), (\mathbb{R}, +), (\mathbb{C}, +)$ sind abelsche Gruppen. Ebenso $(\mathcal{K}, +)$ wenn $(\mathcal{K}, +, \cdot)$ ein Körper ist.
	\item $\mathbb{Q}^* = \mathbb{Q} \setminus \{0\}$, $\mathbb{R}^*, \mathbb{C}^*, \mathcal{K}^* = \mathcal{K} \setminus \{0\}$, mit $1 = e$ als Einselemnent, sind abelsche Gruppen
	\item $GL_n(R) = \{ x \in Mat_n(R) \vert \text{det}(x) \neq 0\}$ mit $R$ beliebiger Körper, invertierbare Matrizen über $R$, $SL_n(R)  = \{ x \in GL_n(R) \vert \text{det }x = 1 \in R \}$, ($1_n = $ Einheitsmatrix $)$ sind Gruppen bezüglich der Matrixmultiplikation, mit Einselement jeweils $1_n$, allerdings für $n > 1$ \underline{nicht} abelsch.
	\item $\mathbb{N}_0 = \mathbb{N} \cup 0$, $\mathbb{N} = \{1,2,3,\dots\}$. Sowohl $(\mathbb{N}_0, +)$ als auch $(\mathbb{N}, \cdot)$ sind Monoide aber keine Gruppen $(2x = 1$ unlösbar$)$.
	\item $Mat_n(R)$ ($R$ Körper) ist ein Monoid $(\cdot = $ Multiplikation, $ e = 1_n)$.
	\item $A \in Mat_n(\mathbb{Z}), L_A = \{ x \in {\mathbb{N}_0}^n \vert xA = 0\}$ ist ein Monoid mit Nullvektor als Einselement.

Notation: $R$ Ring, dann $R^n = \{(r_1, \dots, r_n) \vert r_i \in R\}$ (n-Tupel).
	\item Symmetrische Gruppen: Sei $X$ beliebige Menge. $Sym(X) \coloneqq \{ f : X \to X \vert f$ Bijektion $\}$ ist eine Gruppe mit Verknüpfung von Abbildungen als \glqq Multiplikation\grqq . $(f,g \in Sym(X): f \circ g: X \to X$ Bijektion!$)$, mit $id : X \to X$ als Einselement.
	
	
Rekapitulation der Leibnitz-Formel: $K$- Körper: $a_{ij} = A \in
Mat_n(K)$. 

$$\det(A) = \sum_{\sigma \in S_n} sgn(\sigma)
\prod_{i = 1}^{n} a_{i \sigma(i)}.$$

% TODO: kleiner Exkurs Leibnitz-Formel.

Wichtiger Spezialfall: $X = \{ 1,2, \dots, n \}, n \in \mathbb{N}$. Setze $S_n = Sym(X)$\glqq\ , symmetrische Gruppe vom Grad $n$\grqq . (Dies erlaubt es, dass jede endliche Gruppe als Unterobjekt verstanden werden kann) Ordnung $\vert S_n \vert = n ! $. Nicht abelsch falls $n > 2$.\bigskip

$f \in S_n:$

Matrixschreibweise $\begin{pmatrix}
1 & 2 & \dots & n\\
f(1) & f(2) & \dots & f(n)
\end{pmatrix}$ 

oder Zykelschreibweise: $\left(1, f(1), f(f(1)) \dots \right),\; (a, f(a), f^2(a) \dots), \underbrace{(b, f(b), f^2(b) \dots)}_{\text{\glqq Zykel\grqq}}$ für $a \notin$ \menge{f^n(1)}{n = \{1,2,3,\dots\}} $\subseteq \{1,\dots,n\}$ und $b \notin$ \menge{f^n(1)}{dots} $\cup$ \menge{f^n(a)}{n \in \mathbb{N}}

Konvention: Zykel der Länge 1 weg. \bigskip

$\begin{pmatrix}
1 & 2 & 3\\
1 & 2 & 3
\end{pmatrix} \Leftrightarrow (1)(2)(3)$ (A)

$\begin{pmatrix}
1 & 2 & 3\\
2 & 1 & 3
\end{pmatrix} \Leftrightarrow (12)(3)$ (B)

$\begin{pmatrix}
1 & 2 & 3\\
1 & 3 & 2 
\end{pmatrix} \Leftrightarrow (123)$ (C)

\dots

\item Sei $X$ eine beliebige Menge und $G$ eine Gruppe. Dann ist $G^X \coloneqq Abb(X,G) = \{q : X \to G\}$ mit der folgenden Verknüpfung eine Gruppe: 

Gegeben $\varphi, \psi \in G^X$, definiere für $x \in X \; \phi \circ \psi \coloneqq \phi(x) \cdot \psi(x) \in G$ Dies nennt sich \glqq komponentenweise Multiplikation\grqq .

\item Sei $X$ eine beliebige Menge, $\{G_x\}_{x \in X}$, Familie von Gruppen. Dann ist $\prod_{x \in X} G_x =$ \menge{(g_x)_{x \in X}}{\forall x : g_x \in G_x} mit der Verknüpfung $(g_x)_{x \in X} \cdot (h_x)_{x \in X} \coloneqq (g_x \cdot h_x)_{x \in X}$ -- Produkt der Gruppen $G_x, x \in X$.
\end{enumerate}\bigskip

\subsection{Untergruppen und Homomorphismen}

\begin{defi} Sei $G$ ein Monoid, $H \subseteqq G$ Teilmenge. $H$ heißt \emph{Untermonoid} von G, falls
\begin{itemize}
\item $e \in H$
\item $a,b \in H \Rightarrow ab \in H$
\end{itemize}

Ist $G$ eine Gruppe, so heißt $H$ \emph{Untergruppe}, falls zusätzlich
\begin{itemize}
\item $a \in H \Rightarrow a^{-1} \in H$
\end{itemize}.
\end{defi} Schreibe gegebenenfalls \glqq $H \leqslant G$\grqq\ oder \glqq $H < G$\grqq. Schreibe \glqq $H \lneq G$\grqq\ für Untergruppen $H \neq G$. \bigskip

\textbf{Beispiel (1.5):} \begin{enumerate}
\item $G$ Gruppe $\Rightarrow H = G \leqslant G$, $\{e\} = G$ heißen \emph{triviale Untergruppen}.
\item $G = (\mathbb{Z}, +), m \in \mathbb{Z}$. $H = m \mathbb{Z} =$ \menge{mx}{x \in \mathbb{Z}} $\leqslant \mathbb{Z}$.

	\begin{enumerate}
		\item @1: $0 = m0 \in H$
		\item @2: $mx (\in H) + my (\in H) = m (x + y) \in H$, $x,y \in \mathbb{Z}$
		\item @3: Inverses von $mx$ ist $-mx (mx + (-mx) = 0 = e_\mathbb{Z})$.
	\end{enumerate}

Tatsache (Beweis später): \underline{Jede} Untergruppe von $Z$ ist von der Form $m Z$. $\mathbb{Z} = (-1)Z = 1Z, 0 \mathbb{Z} = e_\mathbb{Z}$. Schreibe $(m) = m \mathbb{Z}$.

Echte Untergruppen: $\{A,B\}, \{A,D,E\}, \{A,C\}, \{A,F\}$.

\item $G$ Gruppe, $g \in G$, $H \coloneqq < g > \coloneqq$ \menge{g^n}{n \in \mathbb{Z}} $\leqslant G$ (!!). %ausruf über leqslant

\begin{enumerate}
\item % checkmark
\item $g^n \cdot g^m = g^{n + m}$ %checkmark
\item $(g^n)^{-1} = g^{-n}$ % checkmark
\end{enumerate}

\glqq Die von $g$ erzeugte (zyklische) Untergruppe\grqq\ $=$ die kleinste Untergruppe von $G$, die $g$ enthält (braucht $g$, $g^2$, $g^3$, \dots , $g^{-1}$, \dots).

\item $SL_n(K) \leqslant GL_n(K)$, $K$ Körper

LHS = \menge{x \in Mat_n(K)}{det(x) = 1}
RHS = \menge{x \in Mat_n(K)}{det(x) \neq 0}

Für alle \glqq vernünftigen\grqq\ Körper und alle $n > 1$ ist das eine nicht-triviale Teilmenge.  
\end{enumerate}

\begin{defi} (1.6) Seien $G, G'$ Monoide, mit Einselementen $e \in G$ und $e' \in G'$. Ein \emph{Monoidhomomorphismus} ist eine Abbildung $\varphi : G \to G'$, derart, dass \begin{itemize}
\item $\varphi(e) = e'$
\item $\forall a, b \in G: \varphi(a \cdot b) = \varphi (a) \cdot \varphi (b)$.
\end{itemize}
Sind $G, G'$ Gruppen, spricht man von einem \emph{Gruppenhomomorphismus} (oder oft einfach nur Homomorphismus).
\end{defi}

\textbf{Bemerkung:} Sei $\varphi : G \to G'$ ein Gruppenhomomorphismus, dann gilt

(Übungsaufgaben:)
\begin{enumerate}
\item $\forall a \in G: (\varphi (a))^{-1} = \varphi (a^{-1})$. Nach der zweiten Eigenschaft von oben gilt $\varphi(a^{-1}) \cdot \varphi(a) = \varphi(a^{-1} \cdot a) = \varphi(e) = e$

\item $\ker (\varphi) \coloneqq$ \menge{g \in G}{\varphi(g) = e'} $\leqslant G$.

\item $img (\varphi) =$ \menge{\varphi(g)}{g \in G}  $\leqslant G'$
\end{enumerate} 

\begin{defi} (1.7) Ein Gruppenhomomorphismus $\varphi : G \to G'$ heißt
\begin{itemize}
	\item \emph{Monomorphismus}, falls er injektiv ist. $(\Leftrightarrow \ker (\varphi) = \{e\})$,
	\item \emph{Epimorphismus}, falls er surjektiv ist. $(\Leftrightarrow im(\varphi) = G')$,
	\item \emph{Isomorphismus}, falls er sowohl ein Epimorphismus als auch ein Monomorphismus ist,
	\item \emph{Endomorphismus} von $G$, falls $G' = G$,
	\item \emph{Automorphismus} von $G$, falls $G' = G$ und $\varphi$ ein Isomorphismus ist.
\end{itemize}
\end{defi}

Sprechweise: Gegeben ein Isomorphismus heißen $G$ und $G'$ \emph{isomorph zu einander}. Man schreibt $G \cong G'$. Beispiel: $G = \mathbb{Z}, H = 2 \mathbb{Z} \leqslant G$.

Behauptung: $G \to G$, $g \mapsto 2g (= g + g)$ ist Isomorphismus. Also schreibt man $\mathbb{Z} \cong 2\mathbb{Z}$. Isomorphie ist transitiv $(G \cong G'), (G' \cong G'') \Rightarrow G \cong G''$.\bigskip

\textbf{Beispiele (1.8):}
\begin{enumerate}
\item Sei $G$ ein Monoid, $g \in G$, dann ist $\varphi : \mathbb{N}_0 \to G, n \mapsto g^n$ ein Monoidhomomorphismus. (Übungsaufgabe!) $\varphi$ ist sozusagen bestimmt durch $\varphi(1)$. 

\item Ist $G$ sogar eine Gruppe, dann bestimmt $n \mapsto g^n$ einen Gruppenhomomorphismus: $\varphi : \mathbb{Z} \to G$. $im(\varphi) = $ \menge{g^n}{n \in Z} $= < g >$.

Übungsaufgabe: Wenn $\vert <g> \vert = \infty$, dann ist $\phi : \mathbb{Z} \to im(\varphi)$ ein Isomorphismus.

\item Sei $G$ eine Gruppe, $g \in G$. Dann ist $\psi_g \coloneqq G \to G, h \mapsto g h g^{-1}$ ein Gruppen\underline{automorphismus} (Übungsaufgabe: Checken und was ist das Inverse zu $\psi_g$?).

$\psi_g(h \cdot h') = g h h' g^{-1} = gheh'g^{-1} = (ghg^{-1})(gh'g^{-1}) = \psi_g(h) \psi_g(h')$. Dies nennt sich \glqq Konjugation mit $g$\grqq .

Prüfen: $Aut(G) =$ \menge{\psi : G \to G}{ \phi Automorphismus} bildet Gruppe unter Verknüpfung.

\menge{\phi_g}{g \in G} $< Aut(G)$ = "Innere Automorphismen von $G$".
\end{enumerate}

\subsection{Nebenklassen}

\begin{defi}
Sei $G$ eine Gruppe, $H \leqslant G$. Eine Menge der folgenden Form: $gH = $\menge{g \cdot H}{h \in H}  $\subseteq G, g \in G$ heißt \emph{Linksnebenklasse} (LNK) (left coset) von $H$ in $G$.

Dementsprechend: $Hg = $ \menge{h \cdot g}{h \in H} heißt Rechtsnebenklasse (RNK) von $H$ in $G$.
\end{defi}

Nebenklassen \glqq pflastern\grqq\ die Gruppe.

\textbf{Lemma (1.10):} Seien $gH$ und $g'H$, mit $g, g' \in H$ Linksnebenklassen von $H \leqslant G$. Dann sind Äquivalent:
\begin{enumerate}
\item $gH = g'H$
\item $gH \cap g'H \neq \varnothing$
\item $g \in g'H$
\item $g'^{-1} g \in H$
\end{enumerate}

\textbf{Beweis:}
\begin{enumerate}
\item klar: $gH \ni ge = g \Rightarrow gH \neq \varnothing$
\item $gh \in g'H$ für ein $h \in H$. $\Rightarrow \exists h' \in H: gh = g' h' \Leftrightarrow g \underbrace{h h^{-1}}_{e} = g'\underbrace{h'h^{-1}}_{\in H} \Leftrightarrow g \in g'H$.
\item $g \in g'H \Leftrightarrow \exists h \in H: g = g'h \Rightarrow (g')^{-1} \cdot g = h \in H.$
\item $(g')^{-1} g \in H$, etwa $(g')^{-1}g = h \in H \Rightarrow g = g' \cdot h$. $gH = $\menge{g\tilde{h}}{\tilde{h} \in H} $=$ \menge{g' \cdot (h \tilde{h})}{\tilde{h}} $= g'H\qquad\square$
\end{enumerate}

\textbf{Satz (1.11):} Je zwei Linksnebenklassen von $H$ in $G$ sind in Bijektion zueinander. Verschiedene LNK sind disjunkt. Insbesondere ist $G$ disjunkte Vereiningung der Linksnebenklassen von $H$ in $G$.
\textbf{Beweis:} @Bijektion: Gegeben $gH$, $g'H$, $g, g' \in G$. \\
$gH \underbrace{\cong}_{(bijektiv)} H$. ($gH \cong eH = H \cong g'H$)

Es reicht zu zeigen: $H \to gH, h \mapsto gh$ ist Bijektiv. $gh = gh' \leftrightarrow g^{-1}gh = g^{-1}gh' \leftrightarrow h = h'$. $\square$

\begin{defi} (1.12)
Sei $G$ eine Gruppe, $H \leqslant G$. Schreibe $G / H = $ \menge{gH}{g \in G} für die Menge der Linksnebenklassen von $H$ in $G$ und $H \ G$ = \menge{Hg}{g \in H} für die Menge der Rechtsnebenklassen von $H$ in $G$.
\end{defi}

\textbf{Bemerkung:} $G / H \to H \ G, gH \mapsto Hg$ ist Bijektion.

ÜA: $(gH = g'H \Leftrightarrow$ %Lemma 1.10 (4).
$g \cdot (g')^{-1} \in H \Leftrightarrow Hg = Hg')$

\begin{defi} (1.13)
Setze $\vert G : H\vert (= [G:H] = (G:H)) = \vert G / A\vert =$ %siehe Bemerkung
$= \vert H \ G\vert$ genannt der \emph{Index} von $H$ in $G$. 
\end{defi}

\textbf{Informell:} $\vert G : H\vert$ \glqq inverse Dichte\grqq\ von $H$ in $G$. $\frac{1}{\vert G : H\vert} = $\glqq $ P(g \in H)$\grqq .

z.B.: $G = \mathbb{Z}, H = m \mathbb{Z}$. $\rightarrow \vert G : H\vert
 = m$.
 
\textbf{Beispiel (1.14) :} \begin{enumerate}
\item $H = \{ e \}: G/H \cong H \ G \cong G$ %congs sind Bijektionen
$\Rightarrow \vert G : H\vert = \vert G \vert$.
\item $H = G: G/G = G\\G = \{\cdot\}$ % einelementige Menge
$\Rightarrow \vert G : G\vert = 1$.
\item Drittes Beispiel siehe oben.
\end{enumerate}

\textbf{Korrollar (1.5) (Satz von Lagrange):} Sei $G$ eine \underline{endliche} Gruppe, $H \leqslant$. Dann gilt $\vert G\vert = \vert H \vert \cdot \vert G : H \vert$. ($\frac{\vert G\vert}{\vert H \vert} = \vert G : H \vert$).\\ (1.16) Insbesondere teilt $\vert H \vert$ stets $\vert G\vert$!

\textbf{Achtung:} $\exists$ endliche Gruppen $G$ mit der Eigenschaft, dass einige Teiler ihrer Ordnung von von Untergruppen $H$ realisiert werden. (ÜA: Eleganter formulieren.)

z.B: Ist $p$ eine Primzahl, $\vert G\vert = p^e, e \in \mathbb{N}_0$, so heißt $G$ \emph{p-Gruppe.} Aus Langrange folgt: $\forall H \leqslant G, H$ ist p-Gruppe.

\subsection{Normalteiler \& Isomorphiesätze}

\begin{defi}(1.17)
Sei $G$ eine Gruppe, $H \leqslant G$ heißt \emph{Normalteiler} (oder synonym dazu: normale Untergruppe), wenn $\forall g \in G\; gH = Hg.$
Gegebenenfalls heißt $gH$ die von $g$ bestimmte Nebenklasse von $H$ in $G$. Gegebenenfalls schreibe $(H \trianglelefteq G) \Leftrightarrow H \triangleleft G$.
\end{defi}

\textbf{Bemerkung:} $g, H, G$ wie in (1.17): $gH = Hg \Leftrightarrow g H g^{-1} = H$.\\
Um zu verifizieren, ob $H \leqslant G$ ein Normalteiler ist, reicht es zu testen, ob $\forall g. g H g^{-1} \subseteq H$ (*).

In der Tat. Angenommen (*) gilt, so gilt auch $\forall g \in G. gH \subseteq Hg$. Aber es gilt $g^{-1}H{g^{-1}}^{-1} = g^{-1}Hg \subseteq H, $ d.h. $Hg \subseteq gH$

$\Rightarrow gH = Hg$ % Das folgt aus den Zweien oben.

\textbf{Beispiel (1.18):}\begin{enumerate}
\item $\{ e \} \triangleleft G$. ($ \forall g: g \{ e \} = \{ e \} g = \{ g \cdot e \} = \{ g \}$). Ebenso: $G \triangleleft G: gG = G = Gg$.
\item Ist $G$ abelsch, ist \underline{jede} Untergruppe Normalteiler. 
\item $G = S_3$. Die Untergruppen $H$ von $S_3$ die von $\{e\}$ und $S_3$ selbst verschieden sind, sind $\{<(12)>, <(13)>, <(23)>, <(123)>\}$ (Untergruppen der Ordnung 2,2,2 und 3. Lagrange sagt dass es keine anderen geben kann.)
Übungsaufgabe: Von diesen 4 Untergruppen ist nur $<(123)>$ normal.
\item Ist $\varphi : G \rightarrow G'$ Gruppenhomomorphismus, dann ist $ker(\varphi) = \lbrace g \in G | \varphi(g) = e_{G'} \rbrace \lhd G$.\\
      Z.z. $\forall g \in G . g(ker \varphi) g^{-1} \subseteq ker \varphi$. Sei $g k g^{-1}$ mit $k \in ker \varphi$. Reicht zu zeigen: $\varphi(gkg^{-1}) = e_{G'}$. Aber 
      \begin{eqnarray*}
\varphi(gkg^{-1} & = & \varphi(g) \varphi(h) \underbrace{\varphi(g^{-1})}_{= \varphi(g)^{-1}}\\
              & = & \varphi(g) e_{G'} \varphi(g)^{-1}\\
              & = & \varphi(g) \varphi(g)^{-1} = e_{G'}       
      \end{eqnarray*}
\end{enumerate}

Sei $N \lhd G$. Wollen Gruppenstruktur auf $G/N = \lbrace gN | g \in G\rbrace$.\\
Allgemein $X,Y \subseteq G : XY := \lbrace xy | x \in X, y \in Y \rbrace \subseteq G$\\
Definiere $\cdot : G/N \times G/N \rightarrow G/N, (gN, kN) \mapsto ghN$\\
$gN = g'N \Leftrightarrow g(g')^{-1} \in N$\\

Seien also $g' \in G, k' \in G$ mit $gN = g'N, kN = k'N$. Wir wissen, $g(g')^{-1} =: n_1 \in N, k(k')^{-1} =: n_2 \in N$.\\
Zu zeigen: $gkN = g'k'N$, d.h. $gh(g'h')^{-1} \in N$. Nun ist 
\begin{eqnarray*}
gh(g'h')^{-1} &=& g \underbrace{h(h')^{-1}}_{n_2}(g')^{-1}\\
&=& g n_2 (g')^{-1}\\
&=& g n_2 g^{-1} \underbrace{g (g')^{-1}}_{n_1}\\
&=& \underbrace{g n_2 g^{-1}}_{\in N, da N \lhd G} n_1 \in N 
\end{eqnarray*}

Übungsaufgabe: Verifiziere, dass $\cdot$ eine Gruppenoperation ist.

\textbf{Bemerkung:} $N \lhd G$. Die Gruppe $G/N$ (\glqq G modulo N\grqq ) heißt \glqq Faktorengruppe von G nach N\grqq .\\
$\pi : G \rightarrow G/N, g \mapsto gN$ heißt die \glqq natürliche Reduktion\grqq , \glqq natürlicher Homomorphismus\grqq , \glqq natürliche Surjektion\grqq , \glqq Reduktionmodulo N\grqq .\\
$\pi$ ist Epimorphismus. ($g,k \in G : \underbrace{\pi(gh)}_{gh N} = \underbrace{\pi(g) \pi(h)}_{gN \cdot hN}$)

\begin{satz}[1.19]{\glqq Homomorphiesatz\grqq}
 Sei $\varphi: G \rightarrow G'$ Homomorphismus von Gruppen $G$, $G'$ und sei $N \lhd G$ mit $N \leq ker \varphi$. Dann existiert \emph{genau ein} Homomorphismus $\bar{\varphi} : G/N \rightarrow G'$, derart, dass $\varphi = \bar{\varphi} \circ \pi$. *malt ein kommutierendes Diagramm*\\
 Es gilt:
 \begin{itemize}
  \item $im(\bar{\varphi}) = im(\varphi)$
  \item $ker(\bar{\varphi}) = \pi(ker(\varphi)) \lhd G/N$
  \item $ker(\varphi) = \pi^{-1}(ker(\bar{\varphi})) \lhd G$
 \end{itemize}
\end{satz}
\textbf{Beweis:} Eindeutigkeit. Angenommen $\bar{\varphi}$ mit $\varphi = \bar{\varphi} \circ \pi$ existiert $\forall g \in G . \bar{\varphi}(gN) = \bar{\varphi}(\pi(g)) = \varphi(g) \square$\\
Existenz: Gegeben $gN \in G/N$. Definiere $\bar{\varphi}(gN) := \varphi(g)$. Wohldefiniert?\\
In der Tat, seien $g,g' \in G: gN = g'N$, das heißt $g=g'n \exists n \in N$. Zu zeigen: $\varphi(g = \varphi(g')$.\\
$\varphi(g) = \varphi(g'n) = \varphi(g') \varphi(n) = \varphi(g')$, da $N \leq her \varphi$.\\
\smallskip
Homomorphieeigenschaft: Seien $gN, hN \in G/N$.\\
$\bar{\varphi}(gN \cdot hN) = \bar{\varphi}(gh N) = \varphi(gh) = \varphi(g) \varphi(h) = \bar{\varphi}(gN) \cdot \bar{\varphi}(hN)$\\
\begin{enumerate}
 \item $ker \varphi = \pi^{-1}(ker(\bar{\varphi})$, da $\varphi = \bar{\varphi} \circ \pi$
 \item $im(\bar{\varphi}) = im(\varphi), ker \bar{\varphi} = \pi(ker \varphi)$, da $\pi$ \emph{surjektiv} ist.$\square$
\end{enumerate}
\textbf{Beachte:} $\bar{\varphi}$ Monomorphismus gdw. $ker \bar{\varphi} = N$ gdw. $ker{\varphi} = N$\\
Nenne $\bar{\varphi}$ den von $\varphi$ auf $G/N$ \emph{induzierten} Homomorphismus.\\
\begin{korr}[1.20]
Ist $\varphi$ ein Epimorphismus, dann gilt $G/ker(\varphi) \rightarrow G'$ ist Isomorphismus von Gruppen.
\end{korr}
\begin{satz}[1.21]{1. Isomorphiesatz}
Sei $G$ Gruppe, $H \leq G$, $N \lhd G$.
\begin{itemize}
 \item $HN \leq G$ ist Untergruppe (nicht nur Teilmenge). $N \lhd HN$
 \item $H \cap N \lhd H$
 \item $\varphi : {H}/{H \cap N} \rightarrow {HN}/{N}, h(H \cap N) \mapsto h N$ ist Isomorphismus
\end{itemize}
\end{satz}
\textbf{Beweis:}\\
Übungsaufgabe: $G$ Gruppe, $X \subseteq G$: $X \leq G$ gdw. 1. $X \neq \emptyset$, 2. $\forall x,y \in X : x (y)^{-1} \in X$\\
\smallskip
@1: $HN \ni e \cdot e = e$, $\Rightarrow HN \neq \emptyset$\\
$h_1 n_1, h_2 n_2 \in HN, h_i \in H, n_i \in N$ \\
Zu zeigen: $h_1 n_1 (h_1 n_1)^{-1} \in HN$. In der Tat $h_1 n_1 n_2^{-1} h_2^{-1} = \underbrace{h_1 h_2^{-1}}_{\in H} (\overbrace{h_2 \underbrace{n1 n_2^{-1}}_{\in N} h_2^{-1}}^{\in N, da N \lhd G}) \in HN$\\
$N \leq HN, N \lhd G$ impliziert, dass $N \lhd HN$:\\
$N \lhd G \Leftrightarrow \forall g \in G : g N g^{-1} \subseteq N \Rightarrow \forall k \in HN : kN k^{-1} \subseteq N \Leftrightarrow N \unlhd HN$\\
\textbf{Bemerkung:} $(HN)/N nicht \cong H$. Z.B. $HH/H = H/H = \lbrace e \rbrace$\\
Betrachte $\psi: H \rightarrow {HN}/{N}, h \mapsto h e N = hN$ offensichtlich Epimorphismus.\\
$ker \psi = \lbrace h \in H | kN = N \rbrace = H \cap N$ (@2)
\begin{korr}[1.22]
$H/(H\cap N) \rightarrow HN/N$ ist Isomorphismus $\square$
\end{korr}

\begin{satz}[1.22]{2. Isomorphiesatz}
 am Montag...
\end{satz}



%%%%%%%%%%%%%%%%%%%%%%%%%%%%%%%%%%%%%%%%%%%%%%%%%%%%%%%%%%%%%%%%%%%%

\newpage
\section{Übungsaufgaben}

\subsection{13/10/2014}

\textbf{Eindeutigkeit des Neutralen:} Sei $M$ ein Monoid. Zeige, dass $e \in M$ eindeutig ist.\bigskip

Angenommen, es gäbe ein zweites neutrales Element $e'$ mit $e \neq e'$. Dann würde gelten $e = e \cdot e' = e' \rightarrow \bot \quad \square$\bigskip

\textbf{Eindeutigkeit der Inversen:} Sei $M$ ein Monoid. Zeige, dass $a^{-1} \in M$, falls es existiert, eindeutig ist.\bigskip

Angenommen, zu einem $a \in M$ gäbe es zwei inverse Elemente $a', a''$ mit $a' \neq a''$. Dann gilt $a' \cdot a \cdot a''= a' \cdot e = a'$ als auch $(a' \cdot a) \cdot a'' = e \cdot a'' = a''$. Es folgt $a' = a'' \rightarrow \bot \qquad \square$

\subsection{15/10/2014}

\textbf{\underline{Aufgabe:}} Sei $G$ eine Gruppe mit $g \in G$. Zeige: $\psi_g \coloneqq G \to G, h \mapsto g h g^{-1}$ ist ein Gruppenautomorphismus.

\end{document}