\documentclass[10pt,a4paper]{article}
\usepackage[utf8]{inputenc}
\usepackage[german]{babel}
\usepackage{mathtools}
\usepackage{mathtools}
\usepackage{amsmath}
\usepackage{amsfonts}
\usepackage{amssymb}
\usepackage{graphicx}
\usepackage{yfonts}
\usepackage[all]{xy}  % lustige Diagramme
\usepackage[left=2cm,right=2cm,top=2cm,bottom=2cm]{geometry}

\usepackage{multicol} % Multiple columns

% Setting information for document
\author{Jonas Betzendahl}
\date{\today}

% For definitions
\newtheorem{defi}{Definition}
\newtheorem{satz}{Satz}
\newtheorem{korr}{Korrolar}
\newtheorem{lemma}{Lemma}
\newtheorem{prop}{Proposition}

% redifine \em for \emph to use bold instead of italics
\makeatletter
\DeclareRobustCommand{\em}{%
  \@nomath\em \if b\expandafter\@car\f@series\@nil
  \normalfont \else \bfseries \fi}
\makeatother

\begin{document}
\newcommand{\menge}[2]{$\{\,{#1}\,\vert\,{#2}\,\}$}
\parindent0pt

\begin{center}

\Huge \textbf{\underline{Algebra I}} \\\bigskip
\normalsize\normalsize Stand vom \today\\

\begin{multicols}{2}
Dozent: Prof. Dr. Christopher Voll\\
\texttt{voll@math.uni-bielefeld.de   }\bigskip

Autor: Jonas Betzendahl\\
\texttt{jbetzend@techfak.uni-bielefeld.de}\bigskip
\end{multicols}
\end{center}

\tableofcontents
\newpage

\section*{Organisatorisches etc.}

Dozent ist Prof. Dr. Christopher Voll\\
\texttt{voll@math.uni-bielefeld.de}\\
Büro: UHG V5-238, Sprechstunde noch im Flux\bigskip

Vorlesungen finden an Montagen von 08.30 Uhr bis 10.00 Uhr und Mittwochs von 14.15 Uhr bis 15.45 Uhr statt.\bigskip

Es wird darauf hingewiesen, dass die Übungen bei Dr. Doang in Englisch abgehalten werden.\bigskip

Voraussetzung für die Zulassung zur Prüfung sind das Erreichen von mindestens 50 \% der Punkte und mindestens zwei Mal eine aktive Teilnahme an den Übungen (Vorrechnen) abgeleistet zu haben.\bigskip

\begin{tabular}{rlc}
\textbf{Bücher:} & Einführung in die Algebra (F. Lorenz, Spektrum)&\\
        & Algebra 1 (S. Bosch, Springer)&\\
        & Algebra (S. Lang, Springer)&\\
        & Algebra (Hungerford)&\\
        & Algebra (v.d. Waerden) & (Ein Klassiker)\\
        & Algebra (E. Artin) &
\end{tabular}\bigskip

Übungszettel gibt es immer am Mittwoch der Woche n, bearbeiten werden müssen diese bis Mittwoch der Woche n+1 (Abgabe vor der Vorlesung im Postfach des Tutors), besprochen werden sie in der Woche n+2 in den Tutorien.
         
\section*{Algebra - die Kunst, Gleichungen zu lösen (Einführung)}

\underline{Lineare} Algebra:

$a_{11} x_1 + a_{12} x_2 + \dots + a_{nm} x_m = b_1$\\
$ \dots $\\
$a_{n1} x_1 + a_{n2} x_2 + \dots + a_{nm} x_m = b_n$\\

Alle Fragen können in Form $a_{ij}, b_i \in \mathcal{K}$ beantwortet werden.

Verallgemeinerung: $a_0 + a_1 x + a_2 x^2 + \dots + a_n x^n = 0$ (polynom. Gleichung von Grad $n (a_n \neq 0)$.

\glqq Struktur\grqq\ der Lösungen solcher Gleichungen treibt Menschen seit Jahrtausenden um.

Spezialfall: Quadratische Gleichungen

$$ x^2 + bx + c = 0 \Leftrightarrow \frac{-b \pm \sqrt{b^2 - 4c}}{21}$$

durch Wurzeln lösbar. Kubische Gleichungen:

$$ x^3 + b x^2 + cx + d = 0 \Leftrightarrow \dots $$

ebenfalls durch Wurzeln lösbar. Im 16. Jahrhundert wurde bekannt, dass auch quartische Gleichungen $(n = 4)$ durch Wurzelausdrücke lösbar sind.

Was nicht ins Weltbild des 16. Jahrhundert passte: Im 19. Jahrhundert zeigte Abel: Nicht jede quintische Gleichung $(n = 5)$ kann durch Wurzelausdrücke gelöst werden.

Galois: Lösungen von Polynomen sind nicht einfach Mengen ohne Struktur sondern Mengen \emph{mit} Struktur ($\rightarrow$ Gruppentheorie). Auflösbarkeit von $f = 0$ durch Wurzel $\Rightarrow$ Galois-Gruppe($f$) auflösbar.

Ziel der Vorlesung: Einführung in die Sprache der modernen Algebra, sowohl durch Anerkennen der Theorie als auch durch das Praktizieren.

\newpage
\section{Fundamente der Gruppentheorie}

\subsection{Monoide \& Gruppen}

\begin{defi}
	Ein \emph{Monoid} ist eine Menge $M$ zusammen mit einer Verknüpfung $\cdot : M \times M \to M$,
	die die Eigenschaften erfüllt:
	\begin{itemize}
		\item \emph{\texttt{(ASS)}} $\forall a,b,c \in M: (a \cdot b) \cdot c = a \cdot (b \cdot c)$ (Assoziativität)
		\item \emph{\texttt{(NEU)}} $\exists\, e = e_M \in M \; \forall a \in M,\, e \cdot a = a = a \cdot e$ (Existenz
	eines neutralen Elementes)
	\end{itemize} 
\end{defi}

\textbf{Bemerkung:} Die Notation ist oft einfach nur \glqq $ab$\grqq\ statt \glqq $a \cdot b$\grqq , oft auch bei mehreren Monoiden gleichzeitig. Ausgelassen wird immer die passende Verknüpfung. Es wird auch die Schreibweise $\prod_{i=1}^{n} a_i$ für den Ausdruck $a_1 \cdot a_2 \cdot \dots a_n,\; a_i \in M$ verwendet. Weiterhin gelten per Konvention: $\prod_{i=1}^{n} a_i = e$ für $n \leq 0$ und $a^m = \underbrace{a \cdot a \cdot \dots a}_{\text{m-mal, } m \in \mathbb{N}}$.

\textbf{Behauptung:} $e \in M$ ist eindeutig (siehe unten).\bigskip

Sei $a \in M$. Wir nennen $b \in M$ \emph{invers zu} $a$ falls $a \cdot b = b \cdot a = e$ gilt. Falls (!) solch ein $b$ existiert, ist es eindeutig (siehe unten). In diesem Fall ist die Notation oft $a^{-1}$ für $b$: $a \cdot a^{-1} = a^{-1} \cdot a = e$.\bigskip

\begin{defi} Eine \emph{Gruppe} ist ein Monoid $(G, \cdot)$ mit der folgenden Eigenschaft: 

\begin{center}
\emph{\texttt{(INV)}} $\forall a \in G\; \exists\, a^{-1} \in G$ sodass $a \cdot a^{-1} = a^{-1} \cdot a = e$ (Existenz eines inversen Elements)
\end{center}
$G$ heißt \emph{kommutativ} oder synonym dazu \emph{abelsch} falls folgende Eigenschaft gilt:
\begin{center}
\emph{\texttt{(KOM)}} $\forall a, b \in G: a \cdot b = b \cdot a$ (Kommutativität)
\end{center}
\end{defi}

Vektorräume zum Beispiel sind abelsche Gruppen, die interessante Struktur ist hier aber nicht die abelsche Eigenschaft sondern die Multiplikation mit Skalaren, die sich gut mit der Gruppenstruktur verträgt.\bigskip

% Porträt: N.H. Abel 1902-1829 (Norweger).

Die \emph{Ordnung} einer Gruppe ist $(G, \cdot)$ ist die Kardinalität $\vert G \vert$ von $G$.\bigskip

Konvention: \glqq Gruppe $G$\grqq , falls $\cdot$ klar ist. Ist $G$ abelsch, so schreibt man oft $+$ für $\cdot$ (die Monoidverknüpfung) und man redet von \glqq additiver Schreibweise\grqq\ im Gegensatz zu \glqq multiplikativer Schreibweise\grqq . Bei additiver Schreibweise schreibt man oft $0$ für $e$, bzw. bei multiplikativer Schreibweise $1$.\bigskip

\textbf{Beispiele 1.3:}
\begin{enumerate}
	\item $(\mathbb{Z}, +), (\mathbb{Q}, +), (\mathbb{R}, +), (\mathbb{C}, +)$ sind abelsche Gruppen. Ebenso $(\mathcal{K}, +)$ wenn $(\mathcal{K}, +, \cdot)$ ein Körper ist.
	\item $\mathbb{Q}^* = \mathbb{Q} \setminus \{0\}$, $\mathbb{R}^*, \mathbb{C}^*, \mathcal{K}^* = \mathcal{K} \setminus \{0\}$, mit $1 = e$ als Einselemnent, sind abelsche Gruppen
	\item $GL_n(R) = \{ x \in Mat_n(R) \vert \text{det}(x) \neq 0\}$ mit $R$ beliebiger Körper, invertierbare Matrizen über $R$, $SL_n(R)  = \{ x \in GL_n(R) \vert \text{det }x = 1 \in R \}$, ($1_n = $ Einheitsmatrix $)$ sind Gruppen bezüglich der Matrixmultiplikation, mit Einselement jeweils $1_n$, allerdings für $n > 1$ \underline{nicht} abelsch.
	\item $\mathbb{N}_0 = \mathbb{N} \cup 0$, $\mathbb{N} = \{1,2,3,\dots\}$. Sowohl $(\mathbb{N}_0, +)$ als auch $(\mathbb{N}, \cdot)$ sind Monoide aber keine Gruppen $(2x = 1$ unlösbar$)$.
	\item $Mat_n(R)$ ($R$ Körper) ist ein Monoid $(\cdot = $ Multiplikation, $ e = 1_n)$.
	\item $A \in Mat_n(\mathbb{Z}), L_A = \{ x \in {\mathbb{N}_0}^n \vert xA = 0\}$ ist ein Monoid mit Nullvektor als Einselement.

Notation: $R$ Ring, dann $R^n = \{(r_1, \dots, r_n) \vert r_i \in R\}$ (n-Tupel).
	\item Symmetrische Gruppen: Sei $X$ beliebige Menge. $Sym(X) \coloneqq \{ f : X \to X \vert f$ Bijektion $\}$ ist eine Gruppe mit Verknüpfung von Abbildungen als \glqq Multiplikation\grqq . $(f,g \in Sym(X): f \circ g: X \to X$ Bijektion!$)$, mit $id : X \to X$ als Einselement.
	
	
Rekapitulation der Leibnitz-Formel: $K$- Körper: $a_{ij} = A \in
Mat_n(K)$. 

$$\det(A) = \sum_{\sigma \in S_n} sgn(\sigma)
\prod_{i = 1}^{n} a_{i \sigma(i)}.$$

% TODO: kleiner Exkurs Leibnitz-Formel.

Wichtiger Spezialfall: $X = \{ 1,2, \dots, n \}, n \in \mathbb{N}$. Setze $S_n = Sym(X)$\glqq\ , symmetrische Gruppe vom Grad $n$\grqq . (Dies erlaubt es, dass jede endliche Gruppe als Unterobjekt verstanden werden kann) Ordnung $\vert S_n \vert = n ! $. Nicht abelsch falls $n > 2$.\bigskip

$f \in S_n:$

Matrixschreibweise $\begin{pmatrix}
1 & 2 & \dots & n\\
f(1) & f(2) & \dots & f(n)
\end{pmatrix}$ 

oder Zykelschreibweise: $\left(1, f(1), f(f(1)) \dots \right),\; (a, f(a), f^2(a) \dots), \underbrace{(b, f(b), f^2(b) \dots)}_{\text{\glqq Zykel\grqq}}$ für $a \notin$ \menge{f^n(1)}{n = \{1,2,3,\dots\}} $\subseteq \{1,\dots,n\}$ und $b \notin$ \menge{f^n(1)}{dots} $\cup$ \menge{f^n(a)}{n \in \mathbb{N}}

Konvention: Zykel der Länge 1 weg. \bigskip

$\begin{pmatrix}
1 & 2 & 3\\
1 & 2 & 3
\end{pmatrix} \Leftrightarrow (1)(2)(3)$ (A)

$\begin{pmatrix}
1 & 2 & 3\\
2 & 1 & 3
\end{pmatrix} \Leftrightarrow (12)(3)$ (B)

$\begin{pmatrix}
1 & 2 & 3\\
1 & 3 & 2 
\end{pmatrix} \Leftrightarrow (123)$ (C)

\dots

\item Sei $X$ eine beliebige Menge und $G$ eine Gruppe. Dann ist $G^X \coloneqq Abb(X,G) = \{q : X \to G\}$ mit der folgenden Verknüpfung eine Gruppe: 

Gegeben $\varphi, \psi \in G^X$, definiere für $x \in X \; \phi \circ \psi \coloneqq \phi(x) \cdot \psi(x) \in G$ Dies nennt sich \glqq komponentenweise Multiplikation\grqq .

\item Sei $X$ eine beliebige Menge, $\{G_x\}_{x \in X}$, Familie von Gruppen. Dann ist $\prod_{x \in X} G_x =$ \menge{(g_x)_{x \in X}}{\forall x : g_x \in G_x} mit der Verknüpfung $(g_x)_{x \in X} \cdot (h_x)_{x \in X} \coloneqq (g_x \cdot h_x)_{x \in X}$ -- Produkt der Gruppen $G_x, x \in X$.
\end{enumerate}\bigskip

\subsection{Untergruppen und Homomorphismen}

\begin{defi} Sei $G$ ein Monoid, $H \subseteqq G$ Teilmenge. $H$ heißt \emph{Untermonoid} von G, falls
\begin{itemize}
\item $e \in H$
\item $a,b \in H \Rightarrow ab \in H$
\end{itemize}

Ist $G$ eine Gruppe, so heißt $H$ \emph{Untergruppe}, falls zusätzlich
\begin{itemize}
\item $a \in H \Rightarrow a^{-1} \in H$
\end{itemize}.
\end{defi} Schreibe gegebenenfalls \glqq $H \leqslant G$\grqq\ oder \glqq $H < G$\grqq. Schreibe \glqq $H \lneq G$\grqq\ für Untergruppen $H \neq G$. \bigskip

\textbf{Beispiel (1.5):} \begin{enumerate}
\item $G$ Gruppe $\Rightarrow H = G \leqslant G$, $\{e\} = G$ heißen \emph{triviale Untergruppen}.
\item $G = (\mathbb{Z}, +), m \in \mathbb{Z}$. $H = m \mathbb{Z} =$ \menge{mx}{x \in \mathbb{Z}} $\leqslant \mathbb{Z}$.

	\begin{enumerate}
		\item @1: $0 = m0 \in H$
		\item @2: $mx (\in H) + my (\in H) = m (x + y) \in H$, $x,y \in \mathbb{Z}$
		\item @3: Inverses von $mx$ ist $-mx (mx + (-mx) = 0 = e_\mathbb{Z})$.
	\end{enumerate}

Tatsache (Beweis später): \underline{Jede} Untergruppe von $Z$ ist von der Form $m Z$. $\mathbb{Z} = (-1)Z = 1Z, 0 \mathbb{Z} = e_\mathbb{Z}$. Schreibe $(m) = m \mathbb{Z}$.

Echte Untergruppen: $\{A,B\}, \{A,D,E\}, \{A,C\}, \{A,F\}$.

\item $G$ Gruppe, $g \in G$, $H \coloneqq < g > \coloneqq$ \menge{g^n}{n \in \mathbb{Z}} $\leqslant G$ (!!). %ausruf über leqslant

\begin{enumerate}
\item % checkmark
\item $g^n \cdot g^m = g^{n + m}$ %checkmark
\item $(g^n)^{-1} = g^{-n}$ % checkmark
\end{enumerate}

\glqq Die von $g$ erzeugte (zyklische) Untergruppe\grqq\ $=$ die kleinste Untergruppe von $G$, die $g$ enthält (braucht $g$, $g^2$, $g^3$, \dots , $g^{-1}$, \dots).

\item $SL_n(K) \leqslant GL_n(K)$, $K$ Körper

LHS = \menge{x \in Mat_n(K)}{det(x) = 1}
RHS = \menge{x \in Mat_n(K)}{det(x) \neq 0}

Für alle \glqq vernünftigen\grqq\ Körper und alle $n > 1$ ist das eine nicht-triviale Teilmenge.  
\end{enumerate}

\begin{defi} (1.6) Seien $G, G'$ Monoide, mit Einselementen $e \in G$ und $e' \in G'$. Ein \emph{Monoidhomomorphismus} ist eine Abbildung $\varphi : G \to G'$, derart, dass \begin{itemize}
\item $\varphi(e) = e'$
\item $\forall a, b \in G: \varphi(a \cdot b) = \varphi (a) \cdot \varphi (b)$.
\end{itemize}
Sind $G, G'$ Gruppen, spricht man von einem \emph{Gruppenhomomorphismus} (oder oft einfach nur Homomorphismus).
\end{defi}

\textbf{Bemerkung:} Sei $\varphi : G \to G'$ ein Gruppenhomomorphismus, dann gilt

(Übungsaufgaben:)
\begin{enumerate}
\item $\forall a \in G: (\varphi (a))^{-1} = \varphi (a^{-1})$. Nach der zweiten Eigenschaft von oben gilt $\varphi(a^{-1}) \cdot \varphi(a) = \varphi(a^{-1} \cdot a) = \varphi(e) = e$

\item $\ker (\varphi) \coloneqq$ \menge{g \in G}{\varphi(g) = e'} $\leqslant G$.

\item $img (\varphi) =$ \menge{\varphi(g)}{g \in G}  $\leqslant G'$
\end{enumerate} 

\begin{defi} (1.7) Ein Gruppenhomomorphismus $\varphi : G \to G'$ heißt
\begin{itemize}
	\item \emph{Monomorphismus}, falls er injektiv ist. $(\Leftrightarrow \ker (\varphi) = \{e\})$,
	\item \emph{Epimorphismus}, falls er surjektiv ist. $(\Leftrightarrow im(\varphi) = G')$,
	\item \emph{Isomorphismus}, falls er sowohl ein Epimorphismus als auch ein Monomorphismus ist,
	\item \emph{Endomorphismus} von $G$, falls $G' = G$,
	\item \emph{Automorphismus} von $G$, falls $G' = G$ und $\varphi$ ein Isomorphismus ist.
\end{itemize}
\end{defi}

Sprechweise: Gegeben ein Isomorphismus heißen $G$ und $G'$ \emph{isomorph zu einander}. Man schreibt $G \cong G'$. Beispiel: $G = \mathbb{Z}, H = 2 \mathbb{Z} \leqslant G$.

Behauptung: $G \to G$, $g \mapsto 2g (= g + g)$ ist Isomorphismus. Also schreibt man $\mathbb{Z} \cong 2\mathbb{Z}$. Isomorphie ist transitiv $(G \cong G'), (G' \cong G'') \Rightarrow G \cong G''$.\bigskip

\textbf{Beispiele (1.8):}
\begin{enumerate}
\item Sei $G$ ein Monoid, $g \in G$, dann ist $\varphi : \mathbb{N}_0 \to G, n \mapsto g^n$ ein Monoidhomomorphismus. (Übungsaufgabe!) $\varphi$ ist sozusagen bestimmt durch $\varphi(1)$. 

\item Ist $G$ sogar eine Gruppe, dann bestimmt $n \mapsto g^n$ einen Gruppenhomomorphismus: $\varphi : \mathbb{Z} \to G$. $im(\varphi) = $ \menge{g^n}{n \in Z} $= < g >$.

Übungsaufgabe: Wenn $\vert <g> \vert = \infty$, dann ist $\phi : \mathbb{Z} \to im(\varphi)$ ein Isomorphismus.

\item Sei $G$ eine Gruppe, $g \in G$. Dann ist $\psi_g \coloneqq G \to G, h \mapsto g h g^{-1}$ ein Gruppen\underline{automorphismus} (Übungsaufgabe: Checken und was ist das Inverse zu $\psi_g$?).

$\psi_g(h \cdot h') = g h h' g^{-1} = gheh'g^{-1} = (ghg^{-1})(gh'g^{-1}) = \psi_g(h) \psi_g(h')$. Dies nennt sich \glqq Konjugation mit $g$\grqq .

Prüfen: $Aut(G) =$ \menge{\psi : G \to G}{ \phi Automorphismus} bildet Gruppe unter Verknüpfung.

\menge{\phi_g}{g \in G} $< Aut(G)$ = "Innere Automorphismen von $G$".
\end{enumerate}

\subsection{Nebenklassen}

\begin{defi}
Sei $G$ eine Gruppe, $H \leqslant G$. Eine Menge der folgenden Form: $gH = $\menge{g \cdot H}{h \in H}  $\subseteq G, g \in G$ heißt \emph{Linksnebenklasse} (LNK) (left coset) von $H$ in $G$.

Dementsprechend: $Hg = $ \menge{h \cdot g}{h \in H} heißt Rechtsnebenklasse (RNK) von $H$ in $G$.
\end{defi}

Nebenklassen \glqq pflastern\grqq\ die Gruppe.

\textbf{Lemma (1.10):} Seien $gH$ und $g'H$, mit $g, g' \in H$ Linksnebenklassen von $H \leqslant G$. Dann sind Äquivalent:
\begin{enumerate}
\item $gH = g'H$
\item $gH \cap g'H \neq \varnothing$
\item $g \in g'H$
\item $g'^{-1} g \in H$
\end{enumerate}

\textbf{Beweis:}
\begin{enumerate}
\item klar: $gH \ni ge = g \Rightarrow gH \neq \varnothing$
\item $gh \in g'H$ für ein $h \in H$. $\Rightarrow \exists h' \in H: gh = g' h' \Leftrightarrow g \underbrace{h h^{-1}}_{e} = g'\underbrace{h'h^{-1}}_{\in H} \Leftrightarrow g \in g'H$.
\item $g \in g'H \Leftrightarrow \exists h \in H: g = g'h \Rightarrow (g')^{-1} \cdot g = h \in H.$
\item $(g')^{-1} g \in H$, etwa $(g')^{-1}g = h \in H \Rightarrow g = g' \cdot h$. $gH = $\menge{g\tilde{h}}{\tilde{h} \in H} $=$ \menge{g' \cdot (h \tilde{h})}{\tilde{h}} $= g'H\qquad\square$
\end{enumerate}

\textbf{Satz (1.11):} Je zwei Linksnebenklassen von $H$ in $G$ sind in Bijektion zueinander. Verschiedene LNK sind disjunkt. Insbesondere ist $G$ disjunkte Vereiningung der Linksnebenklassen von $H$ in $G$.
\textbf{Beweis:} @Bijektion: Gegeben $gH$, $g'H$, $g, g' \in G$. \\
$gH \underbrace{\cong}_{(bijektiv)} H$. ($gH \cong eH = H \cong g'H$)

Es reicht zu zeigen: $H \to gH, h \mapsto gh$ ist Bijektiv. $gh = gh' \leftrightarrow g^{-1}gh = g^{-1}gh' \leftrightarrow h = h'$. $\square$

\begin{defi} (1.12)
Sei $G$ eine Gruppe, $H \leqslant G$. Schreibe $G / H = $ \menge{gH}{g \in G} für die Menge der Linksnebenklassen von $H$ in $G$ und $H \,\backslash\, G$ = \menge{Hg}{g \in H} für die Menge der Rechtsnebenklassen von $H$ in $G$.
\end{defi}

\textbf{Bemerkung:} $G / H \to H \,\backslash\, G, gH \mapsto Hg$ ist Bijektion.

ÜA: $(gH = g'H \Leftrightarrow$ %Lemma 1.10 (4).
$g \cdot (g')^{-1} \in H \Leftrightarrow Hg = Hg')$

\begin{defi} (1.13)
Setze $\vert G : H\vert (= [G:H] = (G:H)) = \vert G / A\vert =$ %siehe Bemerkung
$= \vert H \,\backslash\, G\vert$ genannt der \emph{Index} von $H$ in $G$. 
\end{defi}

\textbf{Informell:} $\vert G : H\vert$ \glqq inverse Dichte\grqq\ von $H$ in $G$. $\frac{1}{\vert G : H\vert} = $\glqq $ P(g \in H)$\grqq .

z.B.: $G = \mathbb{Z}, H = m \mathbb{Z}$. $\rightarrow \vert G : H\vert
 = m$.
 
\textbf{Beispiel (1.14) :} \begin{enumerate}
\item $H = \{ e \}: G/H \cong H \,\backslash\, G \cong G$ %congs sind Bijektionen
$\Rightarrow \vert G : H\vert = \vert G \vert$.
\item $H = G: G/G = G\\G = \{\cdot\}$ % einelementige Menge
$\Rightarrow \vert G : G\vert = 1$.
\item Drittes Beispiel siehe oben.
\end{enumerate}

\textbf{Korrollar (1.5) (Satz von Lagrange):} Sei $G$ eine \underline{endliche} Gruppe, $H \leqslant$. Dann gilt $\vert G\vert = \vert H \vert \cdot \vert G : H \vert$. ($\frac{\vert G\vert}{\vert H \vert} = \vert G : H \vert$).\\ (1.16) Insbesondere teilt $\vert H \vert$ stets $\vert G\vert$!

\textbf{Achtung:} $\exists$ endliche Gruppen $G$ mit der Eigenschaft, dass einige Teiler ihrer Ordnung von von Untergruppen $H$ realisiert werden. (ÜA: Eleganter formulieren.)

z.B: Ist $p$ eine Primzahl, $\vert G\vert = p^e, e \in \mathbb{N}_0$, so heißt $G$ \emph{p-Gruppe.} Aus Langrange folgt: $\forall H \leqslant G, H$ ist p-Gruppe.

\subsection{Normalteiler \& Isomorphiesätze}

\begin{defi}(1.17)
Sei $G$ eine Gruppe, $H \leqslant G$ heißt \emph{Normalteiler} (oder synonym dazu: normale Untergruppe), wenn $\forall g \in G\; gH = Hg.$
Gegebenenfalls heißt $gH$ die von $g$ bestimmte Nebenklasse von $H$ in $G$. Gegebenenfalls schreibe $(H \trianglelefteq G) \Leftrightarrow H \triangleleft G$.
\end{defi}

\textbf{Bemerkung:} $g, H, G$ wie in (1.17): $gH = Hg \Leftrightarrow g H g^{-1} = H$.\\
Um zu verifizieren, ob $H \leqslant G$ ein Normalteiler ist, reicht es zu testen, ob $\forall g. g H g^{-1} \subseteq H$ (*).

In der Tat. Angenommen (*) gilt, so gilt auch $\forall g \in G. gH \subseteq Hg$. Aber es gilt $g^{-1}H{g^{-1}}^{-1} = g^{-1}Hg \subseteq H, $ d.h. $Hg \subseteq gH$

$\Rightarrow gH = Hg$ % Das folgt aus den Zweien oben.

\textbf{Beispiel (1.18):}\begin{enumerate}
\item $\{ e \} \triangleleft G$. ($ \forall g: g \{ e \} = \{ e \} g = \{ g \cdot e \} = \{ g \}$). Ebenso: $G \triangleleft G: gG = G = Gg$.
\item Ist $G$ abelsch, ist \underline{jede} Untergruppe Normalteiler. 
\item $G = S_3$. Die Untergruppen $H$ von $S_3$ die von $\{e\}$ und $S_3$ selbst verschieden sind, sind $\{<(12)>, <(13)>, <(23)>, <(123)>\}$ (Untergruppen der Ordnung 2,2,2 und 3. Lagrange sagt dass es keine anderen geben kann.)
Übungsaufgabe: Von diesen 4 Untergruppen ist nur $<(123)>$ normal.
\item Ist $\varphi : G \rightarrow G'$ Gruppenhomomorphismus, dann ist $ker(\varphi) = \lbrace g \in G | \varphi(g) = e_{G'} \rbrace \lhd G$.\\
      Z.z. $\forall g \in G . g(ker \varphi) g^{-1} \subseteq ker \varphi$. Sei $g k g^{-1}$ mit $k \in ker \varphi$. Reicht zu zeigen: $\varphi(gkg^{-1}) = e_{G'}$. Aber 
      \begin{eqnarray*}
\varphi(gkg^{-1} & = & \varphi(g) \varphi(h) \underbrace{\varphi(g^{-1})}_{= \varphi(g)^{-1}}\\
              & = & \varphi(g) e_{G'} \varphi(g)^{-1}\\
              & = & \varphi(g) \varphi(g)^{-1} = e_{G'}       
      \end{eqnarray*}
\end{enumerate}

Sei $N \lhd G$. Wollen Gruppenstruktur auf $G/N = \lbrace gN | g \in G\rbrace$.\\
Allgemein $X,Y \subseteq G : XY := \lbrace xy | x \in X, y \in Y \rbrace \subseteq G$\\
Definiere $\cdot : G/N \times G/N \rightarrow G/N, (gN, kN) \mapsto ghN$\\
$gN = g'N \Leftrightarrow g(g')^{-1} \in N$\\

Seien also $g' \in G, k' \in G$ mit $gN = g'N, kN = k'N$. Wir wissen, $g(g')^{-1} =: n_1 \in N, k(k')^{-1} =: n_2 \in N$.\\
Zu zeigen: $gkN = g'k'N$, d.h. $gh(g'h')^{-1} \in N$. Nun ist 
\begin{eqnarray*}
gh(g'h')^{-1} &=& g \underbrace{h(h')^{-1}}_{n_2}(g')^{-1}\\
&=& g n_2 (g')^{-1}\\
&=& g n_2 g^{-1} \underbrace{g (g')^{-1}}_{n_1}\\
&=& \underbrace{g n_2 g^{-1}}_{\in N, da N \lhd G} n_1 \in N 
\end{eqnarray*}

Übungsaufgabe: Verifiziere, dass $\cdot$ eine Gruppenoperation ist.

\textbf{Bemerkung:} $N \lhd G$. Die Gruppe $G/N$ (\glqq G modulo N\grqq ) heißt \glqq Faktorengruppe von G nach N\grqq .\\
$\pi : G \rightarrow G/N, g \mapsto gN$ heißt die \glqq natürliche Reduktion\grqq , \glqq natürlicher Homomorphismus\grqq , \glqq natürliche Surjektion\grqq , \glqq Reduktionmodulo N\grqq .\\
$\pi$ ist Epimorphismus. ($g,k \in G : \underbrace{\pi(gh)}_{gh N} = \underbrace{\pi(g) \pi(h)}_{gN \cdot hN}$)

\begin{satz}[1.19]{\glqq Homomorphiesatz\grqq}
 Sei $\varphi: G \rightarrow G'$ Homomorphismus von Gruppen $G$, $G'$ und sei $N \lhd G$ mit $N \leq ker \varphi$. Dann existiert \emph{genau ein} Homomorphismus $\bar{\varphi} : G/N \rightarrow G'$, derart, dass $\varphi = \bar{\varphi} \circ \pi$. 

\begin{displaymath}
    \xymatrix{          & G \ar[d]^{\varphi} \ar[dl]^{\pi} \\
                G / N \ar[r]^{\bar{\varphi}} & G' \ar[l] }
\end{displaymath} 
 
 Es gilt:
 \begin{itemize}
  \item $im(\bar{\varphi}) = im(\varphi)$
  \item $ker(\bar{\varphi}) = \pi(ker(\varphi)) \lhd G/N$
  \item $ker(\varphi) = \pi^{-1}(ker(\bar{\varphi})) \lhd G$
 \end{itemize}
\end{satz}
\textbf{Beweis:} Eindeutigkeit. Angenommen $\bar{\varphi}$ mit $\varphi = \bar{\varphi} \circ \pi$ existiert $\forall g \in G . \bar{\varphi}(gN) = \bar{\varphi}(\pi(g)) = \varphi(g) \square$\\
Existenz: Gegeben $gN \in G/N$. Definiere $\bar{\varphi}(gN) := \varphi(g)$. Wohldefiniert?\\
In der Tat, seien $g,g' \in G: gN = g'N$, das heißt $g=g'n \exists n \in N$. Zu zeigen: $\varphi(g = \varphi(g')$.\\
$\varphi(g) = \varphi(g'n) = \varphi(g') \varphi(n) = \varphi(g')$, da $N \leq her \varphi$.\\
\smallskip
Homomorphieeigenschaft: Seien $gN, hN \in G/N$.\\
$\bar{\varphi}(gN \cdot hN) = \bar{\varphi}(gh N) = \varphi(gh) = \varphi(g) \varphi(h) = \bar{\varphi}(gN) \cdot \bar{\varphi}(hN)$\\
\begin{enumerate}
 \item $ker \varphi = \pi^{-1}(ker(\bar{\varphi})$, da $\varphi = \bar{\varphi} \circ \pi$
 \item $im(\bar{\varphi}) = im(\varphi), ker \bar{\varphi} = \pi(ker \varphi)$, da $\pi$ \emph{surjektiv} ist.$\square$
\end{enumerate}
\textbf{Beachte:} $\bar{\varphi}$ Monomorphismus gdw. $ker \bar{\varphi} = N$ gdw. $ker{\varphi} = N$\\
Nenne $\bar{\varphi}$ den von $\varphi$ auf $G/N$ \emph{induzierten} Homomorphismus.\\
\begin{korr}[1.20]
Ist $\varphi$ ein Epimorphismus, dann gilt $G/ker(\varphi) \rightarrow G'$ ist Isomorphismus von Gruppen.
\end{korr}
\begin{satz}[1.21]{1. Isomorphiesatz}
Sei $G$ Gruppe, $H \leq G$, $N \lhd G$.
\begin{itemize}
 \item $HN \leq G$ ist Untergruppe (nicht nur Teilmenge). $N \lhd HN$
 \item $H \cap N \lhd H$
 \item $\varphi : {H}/{H \cap N} \rightarrow {HN}/{N}, h(H \cap N) \mapsto h N$ ist Isomorphismus
\end{itemize}
\end{satz}
\textbf{Beweis:}\\
Übungsaufgabe: $G$ Gruppe, $X \subseteq G$: $X \leq G$ gdw. 1. $X \neq \emptyset$, 2. $\forall x,y \in X : x (y)^{-1} \in X$\\
\smallskip
@1: $HN \ni e \cdot e = e$, $\Rightarrow HN \neq \emptyset$\\
$h_1 n_1, h_2 n_2 \in HN, h_i \in H, n_i \in N$ \\
Zu zeigen: $h_1 n_1 (h_1 n_1)^{-1} \in HN$. In der Tat $h_1 n_1 n_2^{-1} h_2^{-1} = \underbrace{h_1 h_2^{-1}}_{\in H} (\overbrace{h_2 \underbrace{n1 n_2^{-1}}_{\in N} h_2^{-1}}^{\in N, da N \lhd G}) \in HN$\\
$N \leq HN, N \lhd G$ impliziert, dass $N \lhd HN$:\\
$N \lhd G \Leftrightarrow \forall g \in G : g N g^{-1} \subseteq N \Rightarrow \forall k \in HN : kN k^{-1} \subseteq N \Leftrightarrow N \unlhd HN$\\
\textbf{Bemerkung:} $(HN)/N nicht \cong H$. Z.B. $HH/H = H/H = \lbrace e \rbrace$\\
Betrachte $\psi: H \rightarrow {HN}/{N}, h \mapsto h e N = hN$ offensichtlich Epimorphismus.\\
$ker \psi = \lbrace h \in H | kN = N \rbrace = H \cap N$ (@2)
\begin{korr}[1.22]
$H/(H\cap N) \rightarrow HN/N$ ist Isomorphismus $\square$
\end{korr}

\begin{displaymath}
    %TODO: Make curved arrows / beautiful
    \xymatrix{ G \ar[d]^{G/H} \ar[dd]^{G/N} \\
               H \ar[d]^{H/N} \\
               N }
\end{displaymath}

\begin{satz}[1.22]{2. Isomorphiesatz}
Sei $G$ eine Gruppe, $N, H \trianglelefteq G, n \leq H$. Dann auch $N \trianglelefteq H, H / N \trianglelefteq G / N$ und $G/N / H/N \simeq G/H$.
\end{satz}
\textbf{Beweis:} Betrachte $\psi: H \hookrightarrow G
\xrightarrow{\pi_n} G/N, h \mapsto k \mapsto kN$,
Homomorphismus. $\ker\psi = N$, insbesondere $N \trianglelefteq H$. Aus dem Homomorphiesatz folgt, dass dies ein Monomorphismus ist. $H/N \to G/N$, konkret: $H/N=$\menge{hN}{h \in H} $\subseteq G/N =$ \menge{gN}{g \in G}.
Identifiziere $H/N$ mit Untergruppe von $G/N$!

Beachte: $H = \ker_{\pi_H}$, $\pi_H: G \to G/H, g \mapsto gH$. $N \leq \ker\pi_H$. Aus dem Homomorphiesatz folgt, dass dies ein Epimorphismus ist. $G/N \to G/H, gN \mapsto gH$ mit Kern $H/N$.

Nach Korollar 1.20: $G/N / H/N \simeq G/H \square$.

\subsection{Erzeugungssysteme \& zyklische Gruppen}

\begin{defi}(1.23)
Sei $G$ eine Gruppe und $X \subseteq G$.

$$\left< X \right> \coloneqq \bigcap_{X \subseteq H \leqslant G} H \leqslant G$$

heißt dann die \emph{von $X$ erzeugte} Untergruppe, die kleinste Untergruppe von $G$, die $X$ enthält. 

\textbf{Spezialfall:} $\left< X \right> = G.$ Nenne $X$ ein \emph{Erzeugendensystem} (generating set / system) von $G$.

Setze $d(G) \coloneqq min$\menge{\vert Y \vert}{Y\text{ EZS von }G}$\in \mathbb{N} \cup \{\infty\}$. $G$ heißt \emph{zyklisch} falls $d(G)=1$. $d(G)$ heißt \emph{(minimale) Erzeugerzahl} von $G$.
\end{defi}

Übungsaufgabe: $d(S_n) = ?$

\textbf{Bemerkungen:}
\begin{enumerate}
\item $X \leqslant G, \left< X \right> = X$.
\item $\left< X \right> =$ \menge{x_1^{\varepsilon _1} x_2^{\varepsilon _2} \dots x_m^{\varepsilon _m}}{m \in \mathbb{N}_0, x_i \in X, \varepsilon_i \in \{ -1, 1\}}

$(x_1^{\varepsilon _1} \dots x_m^{\varepsilon _m})^{-1} = (x_m^{\varepsilon _m} x_{m-1}^{\varepsilon _{m-1}} \dots x_1^{\varepsilon _1})$

\item Für $g \in G \quad <g> =$ \menge{g^n}{n \in \mathbb{Z}} $= \left< \{g\} \right>$. (Notation von letzter Woche)
\item $G$ zyklisch $\Leftrightarrow \exists g \in G: G = < g > \Leftrightarrow \exists$ Epimorphismus: $\varphi : \mathbb{Z} \to G, 1 \mapsto g$.
\end{enumerate}

\textbf{Beispiel: (1.24)} \begin{enumerate}
\item $(\mathbb{Z}, +)$ zyklisch $(\varphi = \texttt{id}_\mathbb{Z})$.
\item Für $m \in \mathbb{Z}: \mathbb{Z} / (m) = \mathbb{Z} / m\mathbb{Z} = \{0 + m\mathbb{Z}, 1 + m\mathbb{Z}, \dots, m-1 + m\mathbb{Z}\}$

(Nebenklassen von $m\mathbb{Z} \trianglelefteq \mathbb{Z}$!) 

Epimorphismus: $\mathbb{Z} \to \mathbb{Z} / m\mathbb{Z}, u \mapsto r(u) + m\mathbb{Z}$

Division mit Rest: $\exists a, r \in \mathbb{Z} . u = a
\cdot m + r, 0 \leqslant r \leqslant m-1$

\textbf{Lemma (1.25):} Sei $H \leqslant \mathbb{Z}$. Dann existiert $m \in \mathbb{N}_0$. $H = m\mathbb{Z}$.
\end{enumerate}

\textbf{Beweis:} Wenn $H = \{0\}$, dann $H = 0 \cdot \mathbb{Z}$. OBdA sei $H \neq \{0\}$. 

Sei $m \in H \setminus \{0\}$ das kleinste positive Element von $H$. Behauptung: $H = m \mathbb{Z}$ \glqq $\supseteq$\grqq klar, da $H \leqslant \mathbb{Z}$. Sei $h \in H . \exists b, r \in \mathbb{Z}: h = b \cdot m + r, 0 \leq r < m \Rightarrow h - bm = r \in H$. Nach Wahl von $m$ ist $r = 0 \Rightarrow \bot$. $\square$

\begin{satz}[1.26]
Sei $G$ eine zyklische Gruppe, dann gilt:

$$ G \simeq \begin{cases}
  \mathbb{Z} \left(= \mathbb{Z} / (0) \right), & \text{falls }\vert G \vert = \infty\\
  \mathbb{Z} / m\mathbb{Z}, & \text{falls }\vert G \vert = m < \infty
\end{cases}$$
\end{satz}
\textbf{Beweis:} Sei $G = \left< g \right>, g \in G$. Dann ist $\varphi: \mathbb{Z} \to G, n \mapsto g^n$ ist ein Epimorphismus. Nach Korrolar 1.20 ist $\mathbb{Z}/(\ker \varphi) \simeq G$. Aus Lemma 1.25 folgt $\ker \varphi = m\mathbb{Z}, m \in \mathbb{N}_0$.

$m=0 \Rightarrow G \simeq \mathbb{Z}$. $m > 0: G \simeq
\mathbb{Z} / (m \mathbb{Z}), \vert G \vert = m$. $\square$

\begin{satz}[1.27]
Sei $G$ zyklisch. (1) Jede Untergruppe von $G$ ist zyklisch. (2) Ist $\varphi G \to G'$ ($G'$ beliebige Gruppe!) ein Homomorphismus, dann sind $\ker \varphi$ und Im$\varphi$ zyklisch
\end{satz}
\textbf{Beweis:} @2: \glqq $\ker\varphi \leqslant G$ zyklisch\grqq\ folgt aus (1). $G = \left< g \right>, g \in G$. Im$\varphi =$ \menge{\varphi(h)}{h \in G}  $=$ \menge{\varphi(g^n)}{n \in \mathbb{Z}} $=$ \menge{\varphi(g)^n}{n \in \mathbb{Z}} $= \left< \varphi(g) \right> \leqslant G'$.

@1: Sei $H \leq G$. Sei $\psi : \mathbb{Z} \to G$ ein Epimorphismus. Betrachte $\underbrace{\psi^{-1}(H)}_{= K} \leqslant \mathbb{Z}$ - zyklisch! Insbesondere ist $\psi(K) = H$ - zyklisch nach (2).

\begin{defi}
Sei $G$ eine Gruppe, $g \in G$. Ordnung von $g \coloneqq \vert \left< g \right> \vert$, geschrieben $\vert g \vert$.
\end{defi}

\begin{satz}[1.29](Kleiner Satz von Fermat)
Sei $G$ eine endliche Gruppe, $g \in G$. Dann teilt $\vert g \vert$ die Ordnung $\vert G \vert$ und es gilt $g^{\vert G \vert} = e$.
\end{satz}
\textbf{Beweis:} Betrachte den Epimorphismus $\varphi: \mathbb{Z} \to \left< g \right> = H \leqslant G$, mit $\ker \varphi = m \mathbb{Z}, m \in \mathbb{N}$.

$\vert G \vert < \infty \Rightarrow \vert H \vert = \vert \left< g \right> \vert = \vert g \vert$ teilt $\vert G \vert$. $H \simeq \mathbb{Z} / m\mathbb{Z} \Rightarrow \vert H \vert = m = \vert g \vert.$

$(g^m)^{\vert G \vert / m} = e^{\vert G \vert / m} = e \square$\bigskip

\section{Fundamente der Ringtheorie}

\textbf{Bis aif weiteres: R kommutativ.}

\begin{defi}(2.1)
Ein \emph{Ring} (mit Einselement \texttt{1}) ist eine Menge $R$ mit $+ : R \times R \to R, (x,y) \mapsto x + y$, $\cdot : R \times R \to R, (x,y) \mapsto x + y$, derart, dass
\begin{enumerate}
\item $(R,+)$ abelsche Gruppe (in "additiver Notation" mit Neutralelement $0 \in R$)
\item $(R, \cdot)$ Monoid
\item Distributivgesetze gelten: $\forall a,b,c \in R: (a+b) \cdot c = a  \cdot c + b \cdot c, c \cdot (a+b) = c \cdot a + c \cdot b$.
\end{enumerate}
$R$ heißt \emph{kommutativ}, falls $\forall a,b \in R: ab = ba$.
\end{defi}

\begin{defi}(2.2)
Ein \emph{Unterring} eines Ringes $(R, +, \cdot)$ ist eine Teilmenge $S$, die bezüglich $+$ und $\cdot$ ein Ring ist. $S$ sei eine additive Untergruppe von $(R,+)$ und ein Untermonoid von $(R,\cdot)$.

Das Paar $S \subset R$ heißt auch \emph{Ringerweiterung}.
\end{defi}

\begin{defi}(2.3)
Sei $R$ ein Ring. $R^* = $ \menge{a \in R}{\exists b \in R}: ab = ba = \texttt{1} - Einheitengruppe (\glqq unit group\grqq) von $R$. $a \in R$ heißt \emph{Einheit}, falls $a \in R^*$.
\begin{itemize}
\item $R$ heißt Schiefkörper (\glqq skew field\grqq) falls $R \neq \{0\} und R^* = R \setminus \{0\}$, d.h. jedes von $0$ verschiedenes Element in $R$ ist invertierbar.

\item Ein \emph{Körper} (\glqq field\grqq) ist ein Schiefkörper, bei dem Multiplikation kommutativ ist: $\forall a,b \in R: ab = ba$.

\item Ein Element $a \in R$ heißt \emph{Nullteiler} (\glqq zeri divisor\grqq) falls $\exists b \in R \setminus \{0\} : ab = 0$ oder $ba = 0$. $(0 \in R$ ist Nullteiler!$)$

\item $R$ heißt \emph{nullteilerfrei}, \underline{\emph{Integritätsbereich}}, \emph{Integritätsring} (\glqq (integral) domain\grqq ) falls $R \neq \{0\}$ und $R \setminus \{0\}$ keine Nullteiler hat.
\end{itemize}
\end{defi}

Übungsaufgabe: $(R*, \cdot)$ ist Gruppe!

\textbf{Bemerkung:} \texttt{Warnung:} $a,b,c \in R, ac = bc \leftrightarrow ac - bc = (a-b)c = 0$. \underline{Wenn} $R$ Integritätsbereich ist und $c \neq 0)$, folgt hierais $a = b$.\bigskip

\textbf{Beispiel:} (2.4): \begin{enumerate}
\item $(\mathbb{Z}, +, \cdot)$ ist ein Ring. $\mathbb{Z}^* = \{1,-1\}$, Nullteiler: $\{0\} \Rightarrow \mathbb{Z}$ ist Integritätsbereich!

\item $R$ kommutativ. Dann ist Mat$_n(R)$ ist Ring mit $1 = E_n$ und der Nullmatrix als Nullelement.

Mat$_n(R)^* =$ \menge{A \in Mat_n(R)}{\exists B \in Mat_n(R): AB = BA = E_n} $=$ \menge{A \in Mat_n(R)}{ det A \underline{\in R^*}} $= GL_n(R) (= GL(n;R))$. (general linear)

Zum Beispiel: $GL_n(\mathbb{Z}) =$ \menge{A \in Mat_n(\mathbb{Z})}{det A \in \{-1,1\}}

$n > 1: Mat_n(R)$ im Allgemeinen \underline{kein Integritätsbereich}: $\exists A: A^m (= A \cdot A^{m -1}) = 0$ für geeignetes $m$. 

\item \underline{Hamiltonische Quaternionen}

$\mathbb{H} = \left< e, i, j, k \right>_\mathbb{R} = \mathbb{R} e \oplus \mathbb{R} i \oplus \mathbb{R} j \oplus \mathbb{R} k = $ \menge{a_1 e + a_2 i + a_3 j + a_4 k}{a_i \in \mathbb{R}}

Multiplikationstabelle:
\begin{tabular}{ c | c c c c}
$\cdot$ & e & i & j & k \\
\hline
e & e & i & j & k\\
i & i & -e & k & -j\\
j & j & -k & -e & i\\
k & k & j & -i & -e\\
\end{tabular}

$(a_1 e + a_2 i + a_3 j + a_4 k) (b_1 e + b_2 i + b_3 j + b_4 k) = a_1 b_1 \underbrace{e \cdot e}_{e} + a_1 b_1 \underbrace{e \cdot i}_{i} + \dots + a_3 b_4 \underbrace{j \cdot k}_{i} + \dots$

Tatsache / Übungsaufgabe: $(\mathbb{H}, +, \cdot)$ ist ein Schiefkörper aber kein Körper.

$\mathbb{H} = \underbrace{\underbrace{\mathbb{R} e}_{\simeq \mathbb{R}} \oplus \mathbb{R} i}_{\simeq \mathbb{C}} \oplus \mathbb{R} j \oplus \mathbb{R} k$

\begin{displaymath}
    \xymatrix
    {
      \mathbb{H} \ar@{-}[d] \simeq_\mathbb{R} \mathbb{R}^4\\
      \mathbb{C} \ar@{-}[d] \simeq_\mathbb{R} \mathbb{R}^2\\
      \mathbb{R}
    }
\end{displaymath} 

Von $\mathbb{R}$ zu $\mathbb{C}$ wird etwas gewonnen (algebraische Abgeschlossenheit, von $\mathbb{C}$ zu $\mathbb{H}$ wird allerdings etwas sehr wichtiges verloren (Kommutativität der Multiplikation). Diese Reihe könnte noch weiter gehen, dann wird es aber irgendwann uninteressant.

\item Sei $X$ eine Menge, $X \neq \varnothing$, $(R_x)_{x \in X}$ eine Familie von Ringen, dann ist $P = \prod_{x \in X} R_x$ ein Ring mit Addition $(r_x) + (s_x) = (r_x + s_x)_x$ und der Multiplikation $(r_x)(s_x) = (r_x \cdot s_x)_x$. 

Null: $(0_x)_x$, Eins: $(\texttt{1}_x)_x$

\textbf{Achtung:} $P$ hat Nullteiler, auch wenn alle $R_x$ nullteilerfrei sind: $\vert X \vert = 2: P = \mathbb{Z} \times \mathbb{Z} = \prod_{x \in X} \mathbb{Z}, a = (1,0), b = (0,1) \Rightarrow a \cdot b = (0,0)$
\end{enumerate}

\begin{defi}(2.5) (Polynomring in einer Variablen)
Sei $R$ ein \underline{kommutativer} Ring. Der \emph{Ring der Polynume in einer Variablen $X$} (Polynomring) mit \emph{Koeffizienten in $R$} ist

$R[X] = \mathbb{R}^{\mathbb{N}_0} =$ \menge{(a_0, a_1, \dots)}{\forall i \in \mathbb{N}_0 : a_i \in \mathbb{R}. \text{Für fast alle } i: a_i = 0} mit Addition $(a_i)_{i \in \mathbb{N}_0} + (b_i)_{i \in \mathbb{N}_0} = (a_i + b_i)_{i \in \mathbb{N_0}} (\in R[x])$.

Multiplikation: $(a_i)_{i \in \mathbb{N}_0} \cdot (b_i)_{i \in \mathbb{N}_0} = (c_i)_{i \in \mathbb{N}_0} \in R[X]$ wobei für $i \in \mathbb{N}_0: c_i \coloneqq \sum_{r+s = i} a_r b_s ) = \sum_{j=0}^{i} a_j b_{i-j}$

Schreibe $f(X) = \sum_{i=0}^{\infty} a_i X^i = \sum_{i = 0}^{N} a_i X^i (n >> 0: a_i = 0$ falls $i \geq N)$ anstelle von $(a_i)_{i \in \mathbb{N}_0} \in R[X], g(X) = \sum_{i=0}^{\infty} b_i X^i$

$f(X) \cdot g(X) = \sum_{i = 0}^{\infty} c_i X^i$

\textbf{Idee:} $R \subset S$ Ringerweiterung, $f(X) \in R[X] : \rightsquigarrow f: S \to S, s \mapsto f(s) = \sum_{i=0}^{\infty} a_i s^i \in S$, die von $f$ induzierte\glqq polynomielle Funktion\grqq .

$R = \mathbb{F}_2 = \{0,1\} : f(X) =) X^2 - X = (0, -1, 1, 0, \dots), g(X) = 0 = (0,0, \dots), \quad \mathbb{F}_2 \to \mathbb{F}_2, s \mapsto s^2 - s = 0$ !

Polynome sind verschieden von Polynomfunktionen!
\end{defi}

\begin{defi}(2.6)
Sei $f = (a_i)_{i \in \mathbb{N}_0} \in R[X], R$ ein kommutativer Ring. Für $i \in \mathbb{N}_0$ ist $a_i$ der \underline{i-te Koeffizient von $f$}. Der \emph{Grad} (\glqq degree\grqq) von $f (\neq 0)$ ist max \menge{i}{a_i \neq 0}, geschrieben $\deg (f)$. $\deg(0) = - \infty$.

Sei $f \neq 0$. Dann ist der Leitkoeffizient von $f$ = $a_{\deg f}$. Ist $a_{\deg f} = 1$, so heißt $f$ \emph{normiert}.  
\end{defi}

\begin{satz}[2.7] (Polynomdivision mit Rest)
Sei $R$ ein kommutativer Ring, $g = (a_i) \in R[X]$, dessen Koeffizient $a_{\deg(g)}$ eine Einheit in $R$ ist. Dann \underline{gibt es} zu jedem $f \in R[X]$ \underline{eindeutig} bestimmte $q, r \in R[X]: f = q \cdot g + r, \deg(r) < deg(g)$.
\end{satz}

\subsection{Ideale, Homomorphismen, Faktorringe}

\begin{defi}(2.8)
Sei $R$ ein Ring. Eine Teilmenge $I \subseteq R$ heißt \emph{Ideal (von $R$)} falls gilt:
\begin{enumerate}
\item $0 \in I$
\item $\forall a, b \in I: a+b \in I$
\item $\forall a \in I, b \in R: ab \in I$
\end{enumerate}
Äquivalent: $I$ additive Untergruppe von $R$, abgeschlossen bezüglich Multiplikation mit Elementen von $R$. Gegebenenfalls schreibe $I \trianglelefteq R$
\end{defi}

Sind $I_1 , I_2 \trianglelefteq R$, so sind:

$I_1 + I_2 = $ \menge{i_1 + i_2}{i_1 \in I_1, i_2 \in I_2} $\trianglelefteq R$.

$I_1 \cdot I_2 = $ \menge{\sum_{\text{endlich}}i_1 i_2}{i_1 \in I_1, i_2 \in I_2} $\trianglelefteq R$.

$I_1 \cap I_2 = $ \menge{i \in R}{i \in I_1 \text{ und } i \in I_2} $\trianglelefteq R$.

Allgemeiner: Gegeben eine Familie $(I_x)_{x \in X}$ von Idealen von $R$ (d.h. $I_x \trianglelefteq R \forall x \in X$), definiere

$\sum_{x \in X} I_x \coloneqq$ \menge{\sum_{x \in X} i_x}{ \forall x \in X: i_x \in I_x; i_x = 0 \text{ für fast alle } x \in X}.

Für $i \in I$ schreibe:

$(i) \coloneqq $ \menge{ib}{ b \in R} $\eqqcolon iR \trianglelefteq$, das von $i$ erzeugte \emph{Hauptideal} (principal ideal).

\underline{Check:}\begin{enumerate}
\item $0 = i \cdot 0 \in (i)$
\item $ib_1 + ib_2 = i(b_1 + b_2) \in (i)$ für $b_i \in R$
\item $(ib)b' = i (bb') \in (i)$ für $b, b' \in R$
\end{enumerate}

\begin{defi}(2.9)\begin{enumerate}
\item Sei $R$ ein Ring und $X$ eine Menge. Sei $(i_x)_{x \in X}$ eine Familie von Ringelementen. Dann heißt $I = \sum_{x \in X} (i_x) \trianglelefteq R$ das von den $i_x$ \emph{erzeugt Ideal}. Das kleinste Ideal, das die Elemente $i_x$ enthält. Die Familie $(i_x)_{x \in X}$ heißt (ein) \emph{Erzeugendensystem}.

\item $I \trianglelefteq R$ heißt \emph{endlich erzeugt}, falls es ein endliches Erzeugendensystem zulässt.

\item $I \trianglelefteq R$ heißt \emph{Hauptideal}, falls es ein Erzeugendensystem der Kardinalität $1$ zulässt. $\exists i \in R: I = (i)$.

\item Ist $R$ ein Integritätsbereich, und ist jedes Ideal von $R$ ein Hauptideal, dann heißt $R$ ein \emph{Hauptidealring} (principal ideal domain).
\end{enumerate}
\end{defi}

\textbf{Proposition (2.10):} $(\mathbb{Z}, +, \cdot )$ ist ein Hauptidealring.\\
\textbf{Beweis:} $\mathbb{Z}$ ist offensichtlich Integritätsbereich. Ideale in $\mathbb{Z}$ sind insbesondere additive Untergruppen. Die Untergruppen von $\mathbb{Z}$ sind alle von der Form $m\mathbb{Z} = (m) = $ \menge{m \cdot n}{n \in \mathbb{Z}} $, m \in \mathbb{N}_0 \quad \square$\bigskip

\textbf{Beispiel (2.11):} \begin{enumerate}
\item Der Ring $\mathbb{Z}[X]$ ist kein Hauptidealring: z.B. $I = (2, X) \trianglelefteq \mathbb{Z}[X]$ (Gegeben Elemente $i_1, \dots , i_n \in R: (i_1, \dots , i_n) \coloneqq \sum_{j=1}^{n} (i_j) \trianglelefteq R$)

$(2) \trianglelefteq \mathbb{Z}[X]: (2) = $ \menge{2 \cdot f}{f \in \mathbb{Z}[X]} = \menge{\sum_{i=0}^\infty a_i x^i}{ \text{fast alle } a_i = 0 , a_i \in (2) \leqslant \mathbb{Z} } 

$(X) = $ \menge{X \cdot f}{f \in \mathbb{Z}[X]} = \menge{\sum_{i=0}^\infty a_i x^i}{ \text{fast alle } a_i = 0 , a_0 = 0 } 

$(2,X) = (2) + (X) = $ \menge{f \in \mathbb{Z}[X]}{f = \sum_{i = 0}^{\infty} a_i x^i, \text{fast alle } a_i = 0 , a_0 = 0 } 

\textbf{Behauptung:} $(2,X)$ ist kein Hauptideal (ÜA!)

\item Ist $R$ ein Körper, so gibt es \emph{nur} die Ideale $(0) = \{0\}$ und $(1) = $ \menge {1 \cdot r}{r \in R} $= R$. Sei etwa $x \in I \setminus \{0\}$. Dann ist $x \cdot x^{-1} = 1 \in I$, das heißt $I = (1) = R$.

%% FRAGE: Angenommen R Ring, R /= {0}. Einzige Ideale (0) und R. Ist dann R ein Körper???
\end{enumerate}

\begin{defi}(2.12)
Seien $R, R'$ Ringe. Eine Abbildung $\varphi : R \to R'$ heißt \emph{Ringhomomorphismus}, falls \begin{enumerate}
\item $\forall a,b \in R: \varphi(a + b) = \varphi(a) + \varphi(b)$
\item $\varphi(1) = 1, \forall a,b \in R: \varphi(ab) = \varphi(a) \varphi(b)$.
\end{enumerate}
(d.h. $\varphi$ sei Homomorphismus abelscher Gruppen und Monoiden).

Die in Definition 1.7 eingeführten Begriffe (Epi-, Mono-, Iso-, Endo- und Automorphismus) existieren sinngemäß auch für Ringe ((Schief-)Körper).
\end{defi} 

\textbf{Bemerkung:} Sei $\varphi : R \to R'$ ein Ringhomomorphismus.\begin{enumerate} 
\item Offensichtlich ist eine Komposition von Ringhomomorphismen wieder ein Ringhomomorphismen. 
\item $\ker \varphi \trianglelefteq R \quad (1) \varphi(0) = 0, 2) a,b \in \ker \varphi: \varphi(a+b) = \varphi(a) + \varphi(b) = 0, 3) a \in \ker \varphi, b \in R: \varphi (ab) = \varphi(a) \varphi(b) = 0 \cdot \varphi(b) = 0)$

im$\varphi \leqslant R$ Unterring, i.A. kein Ideal!

\item $R^* =$\menge{a \in R}{ \exists b \in R: ab = ba = 1} -- Einheitengruppe.

$\varphi$ induziert einen Gruppenhomomorphismus: $\varphi^*: R^* \to {R'}^*, a \mapsto \varphi(a)$.

\item Ist $R$ ein Körper, $R' \neq \{0\}$. Dann ist $\varphi$ injektiv. (Angenommen $\varphi$ ist nicht injektiv. Dann ist $\ker \varphi \neq \{0\} \trianglelefteq R$, d.h. $\ker \varphi = R \Rightarrow \bot$)
\end{enumerate}

Sei $I \trianglelefteq R, R$ ein Ring. \underline{Ziel:} Definiere auf $R / I = $ \menge{a + I}{ a \in R} eine Ringstruktur.

\textbf{Addition:} $\forall a,b \in R: (a+I) + (b+I) = (a+b)+I$

\textbf{Multiplikation:} $\forall a,b \in R: (a+I) \cdot (b+I) = (a\cdot b)+I$

Dies ist wohldefiniert! In der Tat, seien $a', b' \in R$ mit $a+I = a' + I, b+I = b' + I$. Das heißt, dass $\exists i_a, i_b \in I: a = a' + i_a, b = b' + i_b.$ 

Zu zeigen: $a \cdot b + I = a' \cdot b' + I$.

$ab+I = (a' + i_a) (b' + i_b) + I = a'b' + \underbrace{a' \cdot i_b}_{\in I \text{ falls kommutativ}} + \underbrace{i_a b'}_{\in I} + \underbrace{i_a i_b}_{\in I} + I = a'b' + I$, da $I \trianglelefteq$

\textbf{ÜA:} $(R/I, +, \cdot)$ ist ein Ring und $\bar{\varphi}: R \to R/I, a \mapsto a + I$ ist ein surjektiver Ringhomomorphismus.

\textbf{Wiederholung:}\\
$I \triangleleft R: R/I = \lbrace a + I | a \in R \rbrace$. $R/I$ Faktorring \glqq R modulo I\grqq \\
$\pi : R \rightarrow R/I, a \mapsto a+I$ surjektiver Ringhomomorphismus
\begin{satz}[2.13]
 Sei $\varphi: R \rightarrow R'$ Ringhomomorphismus und $I \trianglelefteq R$ derart, $I \subseteq \ker\varphi$. Dann existirt ein eindeutiger Ringhomomorphismus $\bar\varphi : R/I \rightarrow R'$ derart, dass
 $$\bar \varphi \circ \pi = \phi$$
 und \begin{itemize}
      \item $im \varphi = im \bar \varphi$
      \item $\ker \bar \varphi = \pi(\ker \varphi)$
      \item $\ker \varphi = \pi^{-1}(\ker \bar \varphi)$
     \end{itemize}

Insbesondere ist $\bar \varphi$ injektiv gdw. $I = ker \varphi$
\end{satz}

\begin{korr}[2.14]
 Ist $\varphi : R \rightarrow R'$ surjektiver Ringhomomorphismus (mit $I = \ker \varphi$), dann ist 
 $$\bar \varphi : R/(\ker \phi) \rightarrow R'$$
 Ringisomorphismus.
\end{korr}
Isomorphisätze 1.20 und 1.21 haben Analoga in Ringtheorie (\glqq Gruppe\grqq $leftrightarrow$ \glqq Ring\grqq , \glqq Normalteiler\grqq $leftrightarrow$ \glqq Ideal \grqq ).\\
\medskip
Motivation: $I \triangleleft R \leadsto R/I$. \textbf{Ziel:} Gegeben ringtheoretische Eigenschaft P: \glqq Was heisst es für I, dass $R/I$ Eigenschaft P hat?\grqq .\\
\begin{defi}[2.15]
 Sei $R$ Ring (kommutativ mit 1), $I \triangleleft R$, $I \neq R$. I heisst
 \begin{itemize}
  \item Primideal (prim) in R, falls gilt: $\forall a,b \in R: ab \in I \Rightarrow a \in I \textrm{ oder } b \in I$
  \item Maximalideal (maximal) in R, falls: $J \trianglelefteq R, I \subseteq J \Rightarrow I = J \textrm{ oder } J = R$ (informell: Es gibt kein echtes Ideal zwischen I und R. Aber vorsicht: nicht eindeutig!)
 \end{itemize}
\end{defi}
In der Literatur: \textfrak{p} Prim, \textfrak{m} Maxi
\begin{lemma}[2.16]
 Seit $R \neq \lbrace 0 \rbrace$ Ring. Das Nullideal $\lbrace 0 \rbrace$ ist
 \begin{itemize}
  \item prim gdw. $R$ IB ist
  \item max gdw $R$ Körper ist
 \end{itemize}
\end{lemma}
\textbf{Beweis:}\\
@1: $(0)$ ist prim: $a\cdot b \in (0) \Leftrightarrow ab = 0 \Rightarrow a \in (0) oder b \in (0) \Leftrightarrow a = 0 \textrm{ oder } b = 0) \Leftrightarrow 0$ ist der einzige Nullteiler $\Leftrightarrow$ $R$ ist IB.\\
@2: Angenommen $(0)$ max in $R$. Zu zeigen: $R$ Körper, das heisst $R^* = R \setminus \lbrace 0 \rbrace$. Sei $a \in R \setminus \lbrace 0 \rbrace$. Es reicht zu zeigen: $(a) = R = (1)$. Klar nach Maximalität des Nullideals $(0)$.\\
Angenommen $R$ ist Körper, So hat $R$ nur die beiden(!) Ideale $(0)$ und $(1) = R$. Damit ist $(0)$ maximal. $\square$\\
\smallskip
Beachte: $R \cong R/(0)$.
\begin{prop}[2.17]
 Sei $R$ ein Ring, $I \triangleleft R$, ($R \neq (0), I \neq R)$. Dann ist $I$
 \begin{enumerate}
  \item prim gdw. $R/I$ ein IB ist.
  \item maximal gdw. $R/I$ ein Körper ist.
 \end{enumerate}
Insbesondere ist jedes \emph{maximale Ideal} prim (Umkehrung gilt nicht!).
\end{prop}
\textbf{Beweis:}\\
$\bar{ } : R \rightarrow R/I; a \in R \bar{a} = a + I$; $x \in R : \bar{x} = \bar{0}$ gdw. $x \in I$\\
@1: $I$ prim gdw. $\forall a,b \in R : (\bar{a}\cdot \bar{b} = \bar{ab} = \bar{0}\Rightarrow \bar{a} = \bar{0} \textrm{ oder } \bar{b} = \bar{0} \Leftrightarrow R/I$ ist IB ($x,y \in R/I, xy = 0 \Rightarrow x =0 \textrm{ oder } y = 0$)\\
@2: Nach Lemma 2.16 reicht zu zeigen: $I$ maximal genau dann wenn in $R/I$ das Nullideal maximal ist. Behauptung folgt aus der Tatsache, dass $\lbrace J \triangleleft R | I \trianglelefteq J \rbrace \leftrightarrow \lbrace \bar J \triangleleft R/I \rbrace$ mit $J \mapsto \bar J = \lbrace \bar j|j\in J\rbrace$ bijektiv ist.$\square$

\begin{korr}[2.18]
 Ideale in $(\mathbb{Z},+,\cdot)$ sind von der Fom $m\mathbb{Z} = (m), m \in \mathbb{N}_0$.\\
 $(m)$ ist \emph{prim} genau dann wenn $m = 0$ oder $m$ Primzahl.\\
 $(m)$ ist \emph{maximal} genau dann wenn $m$ Primzahl.\\
\end{korr}
Seien $a,b \in \mathbb{Z}$ heissen \emph{koprim} (teilerfremd), wenn $ggT(a,b) = 1$: $\exists x,y \in \mathbb{Z}: ax + by = 1$ (Lemma von Bézout).\\
Idealtheoretisch formuliert: $(a) + (b) = (1) = \mathbb{Z}$ - Äquivalent zu Bézouts Lemma.\\
(Variante des) \emph{Chinesichen Restsatzes}: Eine Kongruenz modulo $a \cdot b$ ist lösbar gdw. wenn sie lösbar ist $mod (a)$ und $mod (b)$.\\
Zum Beispiel $a = 2, b = 3$: $X^2 \equiv 5 mod (6)$ lösbar gdw. $X^2 \equiv 5 \equiv 1 mod (2)$ lösbar \emph{und} $X^2 \equiv 5 \equiv 2 mod (3)$. Letzteres ist aber nicht lösbar, also ist die Gleichung $mod (6)$ nicht lösbar.\\
$R$ Ring $I_1, I_2 \trianglelefteq R$ koprim $:\Leftrightarrow I_1 + I_2 = R \Leftrightarrow (I_1,I_2) = 1$

\begin{satz}[2.19]{Chinesischer Restsatz (Chinese Remainder Theorem (CRT))}
Sei $R$ kommutativer Ring mit 1, $I_1, \dots, I_n$ paarweise koprime Ideale. Dann ist
$$\varphi : R \rightarrow R/I_1 \times \dots \times R/I_n, a \mapsto (a + I_1, \dots, a + I_n)$$
ein surjektiver Ringhomomorphismus mit $\ker \varphi = \bigcap_{j=1}^n I_j =: S$.\\
Korrolar 1.24: $\bar \varphi : R/S \overset{\tilde{ }}{\rightarrow} R/I_1 \times \dots \times R/I_j$ ist Ringisomorphismus.
\end{satz}
\textbf{Beweis:}\\
Zeige Surjektivität von $\varphi$! Zeige zunächst: $\forall j \in [n] = \lbrace 1, 2,  \dots, n\rbrace: I_j$ und $bigcap_{i \neq j} I_i$ sind koprim, das heisst, $1 \in I_j + \bigcap_{i \neq j} I_i$.\\
Nach Voraussetzung existieren für $i \neq j$ Elemente $a_i \in I_j, a_i' \in I_i$ mit $a_i + a_i' = 1$.\\
$$1 = \prod_{i \neq j} 1 = \prod_{i \neq j} (a_i + a_i') \in I_j + \prod_{i \neq j} I_i \subseteq I_j + \bigcap_{i \neq j} I_i$$.


 TODO: REST IN VORLESUNG VOM 10.11.

\subsection{Primfaktorzerlegung}

\begin{defi}[2.20]
Ein Integritätsbereich $R$ heißt \emph{euklidisch}, falls es eine \emph{euklidische Normfunktion} $n: R\setminus\{0\} \to \mathbb{N}_0$ gibt, derart, dass $\forall a, b \in R$, mit $a \neq 0, \exists q, r : b = q a + r$, mit $r = 0$ oder $n(r) < n(a)$.
\end{defi}

\textbf{Beispiel 2.21:}\begin{enumerate}
\item $\mathbb{Z}$ mit $n : \mathbb{Z} \setminus \{0\} \to \mathbb{N}_0, n(a) \coloneqq \vert a \vert$.
\item Sei $K$ ein Körper, $R=K[X], n : f \mapsto \deg f$.
\end{enumerate}

\begin{satz}[2.22]
Jeder euklidische Ring ist ein Hauptidealring.
\end{satz}
\textbf{Beweis:} Sei $R$ ein euklidischer Ring, $I \triangleleft R, (0) \neq I$. Es ist zu zeigen, dass $I$ ein Hauptideal.

Wähle ein $a \in I \setminus \{0\}$ derart, dass $n(a)$ minimal ist. Es reicht zu zeigen, dass $I = (a).$

\glqq $\supseteq$\grqq\ klar.

\glqq $\subseteq$\grqq\ Sei $b \in I, b = qa + r$. Ist $r = 0$, so ist $b \in (a)$. Ist $r \neq 0$, dann ist $n(r) < n(a)$. $(r = b - qa \in I)$. Widerspruch zur Wahl von $a$. $\square$\bigskip

\textbf{Notation:} Sei $R$ ein Integritätsbereich. $x,y \in R$. Schreibe \glqq $x \mid y$\grqq\ für \glqq $x$ teilt $y$\grqq ; d.h. $\exists c \in R$, $xc = y$; andernfalls \glqq $x \nmid y$\grqq .

\textbf{Lemma [2.23]:} Sei $R$ ein Integritätsbereich. $x,y \in R$. Dann sind äquivalent: \begin{enumerate}
\item $x \mid y$ und $y \mid x$

\item $x$ und $y$ sind assoziiert, d.h. $\exists c \in R^*: y = xc$

\item $(x) = (y)$
\end{enumerate}

\begin{defi}[2.24]
Sei $R$ ein Integritätsbereich. Ein Element $x \in R \setminus (R^* \cup \{0\})$ heißt \begin{itemize}
\item \emph{irreduzibel}, falls $(\forall y,z \in R. x = yz \Rightarrow (y \in R^*$ oder $z \in R^*))$
\item \emph{prim}, falls $(\forall y, z \in R \; x \mid yz \Rightarrow x \mid y$ oder $x \mid z)$.
\end{itemize}
i.a.W. $(x)$ ist Primideal.
\end{defi}

\begin{satz}[2.25]
Sei $R$ ein Hauptidealring, $x \in R \setminus (R^* \cup \{0\})$. Dann sind äquivalent:\begin{enumerate}
\item $(x)$ ist maximal
\item $(x)$ ist Primideal, d.h. $x$ ist prim.
\item $x$ ist irreduzibel.
\end{enumerate}
\end{satz}
\textbf{Beweis:} \begin{itemize}
\item \glqq $1 \Rightarrow 2$\grqq : maximale Ideale sind prim. (Prop 2.17)
\item \glqq $2 \Rightarrow 3$\grqq : $x$ prim. Sei $x = yz$. Da $x$ prim ist, gilt $x \mid y$ oder $x \mid z$. Angenommen $x \mid y$. Es existiert also $c \in R: y = cx$, also $x = czx$. Also $x(1-cz) = 0, x \neq 0, R$ ist Integritätsbereich, d.h. $1 = cz$, d.h. $z \in R^*$.
\item \glqq $3 \Rightarrow 1$\grqq : Sei $x$ irreduzibel. Z.z.: $(x)$ ist maximal. Sei dazu $x \in (x) \subseteq I = (a) \trianglelefteq R$. Wir wissen: $x \in (a)$, es existiert also $c \in R: x = ca$. Da $x$ irreduzibel ist $c \in R^*$ oder $a \in R^*$. Wenn $c \in R^*$, so gilt $(x) = (a)$. Wenn $a \in R^*$, so gilt $(a) = R$. Also ist $(x)$ maximal 
\end{itemize} $\square$

\textbf{Bemerkung:} \glqq $1 \Leftrightarrow 2$\grqq\ gilt in allgemeinen Integritätsbereichen (nicht notwendigerweise Hauptidealringe).\bigskip

\textbf{Beispiel 2.26:} Allgemein: \glqq prim $\Rightarrow$ irreduzibel\grqq . In Hauptidealringen: \glqq prim $\Leftrightarrow$ irreduzibel\grqq . Im Allgemeinen gilt \glqq irreduzibel $\Rightarrow$ prim \grqq\ nicht. Betrachte $\mathbb{Z} \hookrightarrow \mathbb{Z}[\sqrt{-5}] = $ \menge{a + b \sqrt{-5}}{a,b \in \mathbb{Z}}

$g = 3 \cdot 3 = (2 + \sqrt{-5}) (2 - \sqrt{-5}).$ 

\underline{Tatsache:} $3, 2 \pm \sqrt{-5}$ sind irreduzibel, $3$ ist allerdings nicht prim: $3 \mid 3^2$, aber $3 \nmid (2 \pm \sqrt{-5})$

\begin{defi}[2.27]
Sei $R$ ein Integritätsbereich. $a \in R$.\begin{enumerate}
\item Eine \emph{Zerlegung (Faktorisierung) von $a$ in irreduzible Faktoren} ist eine Produktdarstellung der Form $a = e \cdot p_1 \cdot \dots \cdot p_r$, wobei $e \in R^*$, $p_1, \dots, p_r$ irreduzibel.

\item $a$ hat eine \emph{eindeutige} Faktorisierung in irreduzible Faktoren, falls aus $a = e' \cdot p_1' \cdot \dots \cdot p_s'$ $(e' \in R^*,$ irreduzibel Element $p_i')$ folgt, dass $r =s$ und - nach eventueller Umnummerierung - $(p_i) = (p_i')$ für alle $i = 1, \dots, r$.

\item Ein Integritätsbereich, in dem jedes von $0$ verschiedene Element eine \emph{eindeutige} Faktorisierung in irreduzible Elemente hat, heißt \emph{faktoriell} (factorial, unique factorization domain = UFD).
\end{enumerate}
\end{defi}

\begin{prop}[2.28] Sei $R$ ein Integritätsbereich derart, dass jedes $a \in R \setminus \{0\}$ eine Faktorisierung in irreduzible Elemente besitzt. Dann ist äquivalent:

\begin{enumerate}
\item $R$ ist faktoriell 
\item Jedes irreduzible Element von $R$ ist prim.
\end{enumerate}
\end{prop}

\textbf{Beweis:} \glqq $1 \Rightarrow 2$\grqq : Seien $a,b \in R \setminus \{0\}$ mit $a = e_a p_1 \dots p_r, b = e_b q_1 \dots q_s, e_a, e_b \in R^*$, $p_i, q_j$ irreduzible Elemente. Sei $p \in R$ irreduzibel mit $p \mid ab$. $p \mid \underbrace{e_a e_b \cdot p_1 \dots p_r q_1 \dots q_s}_{a \cdot b}$. Da $R$ faktoriell ist, ist $p$ assoziiert zu einem der $p_i$ oder einem der $q_j$ (d.h. $(p) = (q_j)$ oder $(p) = (p_i)$). Also $p \mid a$ oder $p \mid b$.

\glqq $2 \Rightarrow 1$\grqq : Angenommen $a = e p_1 \dots p_r = e' p_1' \dots p_s'$. z.z.: $r =s$ und $(p_i) = (p_i')$ nach Umnummerierung.

$e, e' \in R^*, p_i, p_j$ irreduzibel.

O.B.d.A. $r > 0$. $p_1 \mid e' p_1' \dots p_s'$. Es existiert also ein $j \in [s]: p_1 \mid p_j'$. Das heißt $\exists c \in R: p_j' = P_1 c$, aber $p_i, p_j'$ irreduzibel. Das heißt $c \in R^*$. Und das wiederum heißt $(p_1) = (p_j')$.

$a = e p_1 \dots p_r = e' \cdot \underbrace{c \cdot p_1}_{= p_j'} \cdot p_1' \dots \hat{p_j'} \dots p_s'$. RIB $\Rightarrow$ d.h. $e p_2 p_r = \underbrace{e' c}_{\in R^*} p_1' \dots \hat{p_j'} \dots p_s'$

Rest per Induktion. $\square$\\

\textbf{Bemerkung:} In einem faktoriellen Ring sind alle irreduziblen Elemente prim.\\

\begin{satz}[2.29]
 Jeder Hauptidealring ist faktoriell
\end{satz}
\textbf{Beweis:} Sei $R$ ein Hauptidealring (insbesondere Itegritätsbereich). Nach Proposition 2.28 reicht zu zeigen:
\begin{enumerate}
 \item Jedes Element in $R\setminus ( R^* \cup \lbrace 0 \rbrace )$ hat eine Zerlegung in irreduzible Faktoren
 \item Jedes irreduzible Element in $R$ ist prim.
\end{enumerate}
@1: Sei $r \in R \setminus ( R^* \cup \lbrace 0 \rbrace )$. Ist $r$ reduzibel, so existiert $c_1, r_1 \in R \setminus R^* : r = c_1 r_1$. Ohne Einschränkung ist $r_1$ reduzibel. Ist $r_1$ reduzibiel, so existieren $c_2, r_2 \in R \setminus R^* : r_1 = c_2 r_2$. Ist $r_2$ reduzibel, so existieren $c_3, r_3 \in R \setminus R^* : r_2 = c_3 r_3$. Terminiert dieser Prozess nicht, so existiert eine unendliche Folge von Elementen $r_1, r_2, r_3, ...$ mit $(r_1) \subsetneq (r_2) \subsetneq (r_3) \subsetneq \dots$. Beachte $I:=\bigcup_{i=1}^\infty (r_i) \triangleleft R$.\\
$R$ ist aber Hauptidealring $\Rightarrow r_\infty \in R : i = (r_\infty)$, insbesondere $r_\infty \in I$.\\
$\exists N : r_\infty \in (r_N)$, das heisst $I \subseteq (r_N)$. Widerspruch zur Aussage, dass $(r_N) \subsetneq (r_{N+1}) \subsetneq \dots$. Das heisst Hauptidealringe sind \textbf{noethersch} (Emmy Noether, 1882-1935).\\
@2: Sei $p\in R$ irreduzibel. Zu zeigen: $p$ ist prim. Seien $a,b \in R$ mit $p \mid ab$ und $p \nmid b$. Zu zeigen: $p \mid b$.\\
$I := (a,p) = \lbrace x a + y p | x,y \in R \rbrace = (r)$ für $r \in R$, da $R$ Hauptidealring ist.\\
Es gilt natürlich $r \mid a, r \mid p$, sagen wir $p = rc$ fur ein $c \in R$. Da $p$ irreduzibel ist, dann ist entweder $r\in R^*$ oder $c \in R^*$. Ist $c \in R^*$, so gilt $(p) = (r)$. Widerspruch zu $\underbrace{p\nmid a}_{(p) \nsupseteq (a)}, \underbrace{r \mid a}_{(r) \supseteq (a)}$.\\
Also ist $r \in R^*$ und $I = R$. Es existiert also $x,y \in R : 1 = xa + yp$.\\
$p \mid ab \Rightarrow \exists c' \in R : pc' = ab$.\\
$b = 1 \cdot b = (xa + yb) b = x(ab) + p (yb) = p (c'x + yb) \Rightarrow p \mid b$. $\square$\\
\begin{korr}[2.30]
 \begin{enumerate}
  \item $\mathbb{Z}$ ist faktoriell : $a \in \mathbb{Z} \setminus \lbrace 0 \rbrace : a = \epsilon \prod_{p\in P} p^{v_p(a)}, v_p(a) \in \mathbb{N}_0, v_p(a) = 0$ für fast alle $p\in P$. $P$ Menge aller Primzahlen, $\epsilon \in \lbrace 1,-1\rbrace$. $v_p(a)$ sind eindeutig bestimmt. Dies nennt man auch die \glqq p-adische Bewertung von a\grqq 
  \item $K$ Körper, $f \in K[X] \setminus \lbrace 0\rbrace $. $f = c \cdot \prod_{g \in P} g^{v_g(f)}, c \in K^*$. $P$ Repräsentatensystem der irreduziblen Polynome in $K[X]$ (Z.B. normierte irreduzible Polynome).
 \end{enumerate}
\end{korr}

\begin{defi}[2.31]
 Sei $R$ Integritätsbereich, $a_1, \dots, a_r \in R$, Ein Element $t\in R$ heisst \emph{gemeinsamer Teiler} (gT, common divisor) von $a_1, \dots, a_r$, wenn $t\mid a_i \forall i = 1,\dots,r$.\\
 Ein gemeinsamer Teiler $d$ von $a_1, \dots, a_r$ heisst \emph{größter gemeinsamer Teiler} (ggT, greatest common divisor = gcd), falls $t \in R $ gemeinsamer Teiler von $a_1, \dots a_r \Rightarrow t\mid d$.\\
 ÜA: \emph{falls} ggT $d$ existiert, so ist er eindeutig bis auf Assoziativität. Schreibe ggf. $d = ggT (a_1, \dots, a_r)$.\\
 \textbf{Analog:} $b \in R$ heisst \emph{gemeinsames Vielfaches} (gV, common multiple) von $a_1,\dots,a_r$, falls $a_i \mid b$ für $i = 1,\dots ,r$. Ein gemeinsames Vielfaches $c \in R$ heisst \emph{kleinstes gemeinsames Vielfaches} (kgV, least common multiple) von $a_1, \dots, a_r$, falls $b \in R$ gemeinsames Vielfaches von $a_1, \dots, a_r \Rightarrow c \mid b$.\\
 Wieder: Ggfs ist $c = kgV(a_1,\dots, a_r)$ eindeutig bis auf Assoziativität.
\end{defi}
\begin{prop}[2.32]
 Sei $R$ faktoriell, $P$ Repräsentatensystem der irreduziblen Elemente $a_1,\dots, a_r \in R\setminus \lbrace 0 \rbrace$:
 $$\forall i \in [r] : a_i = \epsilon_i \prod_{p\in P} p^{v_p(a_i)}, \epsilon_i \in R^*, v_p(a_i) \in \mathbb{N_0}, \textrm{ für fast alle } v_p(a_i) = 0$$
 so existieren ggT und kgV von $a_1,...,a_r$ und 
 $$ggT(a_1,\dots,a_r) = \prod_{p\in P} p^{min\lbrace v_p(a_i) | i = 1, \dots, r\rbrace} \in R$$
 $$kgV(a_1,\dots,a_r) = \prod_{p\in P} p^{max\lbrace v_p(a_i) | i = 1, \dots, r\rbrace} \in R$$
\end{prop}
\textbf{Beweis:} $a \mid b \Leftrightarrow \forall p\in P : v_p(a) \le v_p(b)$. $a = \epsilon_a \prod_{p} p^{v_p(a)}, b = \epsilon_b \prod_p p^{v_p(b)} \square$.

\subsection{Lokalisierungen, Quotientenkörper, Satz von Gauß}

Sei $R$ Ring. $S \subseteq R$, multiplikativ abgeschlossen (Z.B. wenn R Integritätsbereich, $S = R \setminus \lbrace 0 \rbrace$ oder $\textfrak{p} \triangleleft R$ prim, sei $S = R \setminus \textfrak{p}$.\\
Setze $M = \lbrace (a,b) | a \in R, b\in S\rbrace $. Definiere Relation
$$(a,b) \sim (a',b') :\Leftrightarrow \exists c \in S : ab'c = a'bc \overset{R \textrm{ IB}}{\Leftrightarrow} ab' = a'b$$
(ÜA: Diese Relation ist eine Äquivalenzrelation). Schreibe $\frac{a}{b}$ für die Äquivalenzklasse, die $(a,b)$ enthält.\\
\textrm{Bemerkung:} $b$ ist hier nicht zwingend eine Einheit.\\
Schreibe $S^{-1}R (= R_S) = \lbrace \frac{a}{b} | a \in R, b \in S \rbrace$.\\
Dies ist ein Ring mit der durch \glqq Bruchrechnung\grqq \ gegebenen Operation $\frac{a}{b} + \frac{c}{d} = \frac{ad+cb}{bd}$. Homomorphismus $\varphi : R \rightarrow S^{-1}R, a \mapsto \frac{a}{1}$ \grqq Lokalisierung von R an S\grqq .\\
\textbf{Wichtiger Spezialfall:} Wenn $R$ Integritätsbereich ist und $S = R \setminus \lbrace 0 \rbrace$.\\
$S^{-1}R = Q(R) = \lbrace \frac{a}{b} | a,b \in R, b \neq 0 \rbrace$ \emph{Quotientenkörper} von R (field of fractions).\\




%%%%%%%%%%%%%%%%%%%%%%%%%%%%%%%%%%%%%%%%%%%%%%%%%%%%%%%%%%%%%%%%%%%%

\newpage
\section{Übungsaufgaben}

\subsection{13/10/2014}

\textbf{Eindeutigkeit des Neutralen:} Sei $M$ ein Monoid. Zeige, dass $e \in M$ eindeutig ist.\bigskip

Angenommen, es gäbe ein zweites neutrales Element $e'$ mit $e \neq e'$. Dann würde gelten $e = e \cdot e' = e' \rightarrow \bot \quad \square$\bigskip

\textbf{Eindeutigkeit der Inversen:} Sei $M$ ein Monoid. Zeige, dass $a^{-1} \in M$, falls es existiert, eindeutig ist.\bigskip

Angenommen, zu einem $a \in M$ gäbe es zwei inverse Elemente $a', a''$ mit $a' \neq a''$. Dann gilt $a' \cdot a \cdot a''= a' \cdot e = a'$ als auch $(a' \cdot a) \cdot a'' = e \cdot a'' = a''$. Es folgt $a' = a'' \rightarrow \bot \qquad \square$

\subsection{15/10/2014}

\textbf{\underline{Aufgabe:}} Sei $G$ eine Gruppe mit $g \in G$. Zeige: $\psi_g \coloneqq G \to G, h \mapsto g h g^{-1}$ ist ein Gruppenautomorphismus.

\end{document}