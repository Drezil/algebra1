%\documentclass[a4paper,10pt]{article}
\documentclass[a4paper,10pt]{scrartcl}

\usepackage{fullpage}
\usepackage[utf8]{inputenc}

\begin{document}
\section{Cheat Sheet Algebra 1}

\begin{tabular}{l|l|l}
 Was & Definition & Beschreibung \\
 \hline \hline
 Homomorphismus & $\varphi : G \rightarrow G'$ & $\varphi(e) = e'$ und $\forall a,b \in G : \varphi(a\cdot b) = \varphi(a) \cdot \varphi(b)$ \\\hline
 Monomorphismus & $\varphi$ injektiver Homomorphismus & $\ker(\varphi) = \lbrace e\rbrace$ \\\hline
 Epimorphismus & $\varphi$ surjektiver Homomorphismus & $im(\varphi) = G'$ \\\hline
 Isomorphismus & $\varphi$ bijektiver Homomorphismus & injektiv und surjektiv \\\hline
 Endomorphismus & $\varphi : G \rightarrow G$ & bleibt in derselben Gruppe \\\hline
 Automorphismus & $\varphi$ bijektiver Endomorphismus & \\\hline
 Linksnebenklasse & $gH \subseteq G$ & Weiter ist äquivalent: $gH = g'H$, $gH \cap g'H = \emptyset$, \\
		  &		     & $g \in g'H$, $g'^{-1}g \in H$ \\\hline
 $G/H$		& $gH | g \in G$ & analog für Rechtsnebenklassen. \\
		&		& Bijektion zwischen LNK und RNK.\\\hline
 $|G : H|$	& $|G/H|=|H\backslash G|$ & Index von H in G \\\hline
 Satz von Lagrange & $|G| = |H| \cdot |G : H|$ & nur für \emph{endliche} G \\\hline
 Normalteiler & $\forall g \in G : gH = Hg$ & $gH$ ist die von $g$ bestimmte Nebenklasse von $H$ in $G$.\\
 $N \triangleleft G$ & mit $H \leq G$& G abelsch $\Rightarrow$ \emph{Jede} Untergruppe ist Normalteiler.\\\hline
\end{tabular}

\end{document}
